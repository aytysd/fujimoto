\documentclass[twocolumn,platex,10pt]{yokou}

\usepackage[dvipdfmx]{graphicx}
% \usepackage{url}

\usepackage{soturon}

% \usepackage{bxpdfver}
% \setpdfversion{1.7} % 出力PDFバージョンを1.7にする


\title{赤外線センサを用いた害獣検出および行動解析{\large -- 通知機構と行動ビジュアライザ --}}
% \title{赤外線センサを用いた害獣検出および行動解析\par -- 通知機構と行動ビジュアライザ --}
\author{藤本 光}
\group{山口賢一研究室}

\begin{document}

\maketitle

\section{研究背景}
% 背景
自然を相手にする農業従事者が抱える問題として獣害がある.
特に夜間は視界が悪く,農業従事者が直接追い払えないなど獣害の中でも対策が難しい.
これまでの対策としては,荒らされた畑や食べられた農作物の痕跡から害獣を判断し,
電気柵やネットの設置や該当害獣の捕獲など,害獣やその行動に合った対策が行われている\cite{org:vermin-taisaku}.
しかし,柵に隙間がある場合やそもそも想定していた害獣でないなど,
原因を特定し適切な対策が取れるまでに大きな手間と時間・費用を要している.
% 目的
そこで我々の研究グループではこれらの問題を解決するために,
害獣が畑へ侵入したことを検出・通知し,その害獣の行動や大きさを推定することで,
農業従事者の獣害対策を効果的に行えるように支援するシステムを提案する.
% 特に本論文ではクラウドサーバを用いて,
% 畑への侵入したという情報および推定して得られた結果を保存するデータベース,
% 畑への侵入情報から農業従事者へ通知するWebアプリケーション,
% 推定された行動や大きさを地図を用いて行動履歴を表示するWebアプリケーション
% の設計を行った.

\section{獣害対策支援システム}
本稿で提案するシステムの全体図を\reff{fig:all-system}に示す.
まず,畑に設置した赤外線センサノード群で害獣の侵入を検知し,
そのデータをワイヤレス通信のLoRa通信を用いてゲートウェイに集約する.
ゲートウェイは集約されたデータをLTE通信によりInternetを経由させて
クラウドサーバにデータを送信する.
クラウドサーバではデータの管理やデータの解析,
農業従事者への通知および解析結果の表示を行う.

本システムは著者を含めた3人で実現させる.
著者はクラウドサーバの解析アルゴリズムを除いた
通知サーバ,
Webアプリ,
データベースの設計・実装を行い,
農業従事者へPush通知と行動履歴表示の実現する.


\myfigures{./images/all-system.pdf}{全体のシステム図}{fig:all-system}{9}


\section{提案システムのクラウドサーバ}
システム全体のうち,本稿で対象とするのはクラウドサーバの解析アルゴリズム以外である,
WebPushを用いた通知サーバ,
Leaflet.jsによる地図を用いたWebアプリ,
センサデータや解析データなどを保存するデータベース
である.
クラウドサーバの処理を説明するためにクラウドサーバのDFD(Data Flow Diagram)を\reff{fig:server-dfd}に示す.
クラウドサーバの処理には
畑のデータを送ってくるゲートウェイ,
本システムに登録して通知や解析結果を受け取る農業従事者,
センサデータや解析データ,農業従事者の登録者情報を保存するデータベースが関わっている.
ゲートウェイから畑で得られた赤外線センサのデータはクラウドサーバ上でデータベースに一時的に保存する.
保存したセンサデータを対象に解析を行い,通知と解析後データの保存を行う.
通知では事前に登録した農業従事者の登録者情報からWebPushを用いたPush通知を行う.
また,得られた解析後データをからLeaflet.jsを用いた行動履歴表示のある地図をWebアプリとして農業従事者へ提供する.

\myfigure{./images/server-dfd.pdf}{クラウドサーバのDFD}{fig:server-dfd}

\section{実験結果}
クラウドサーバがやり取りする畑のゲートウェイに搭載する
LTE無線モジュールによるPOST形式の疎通確認および通信形式の確認,
クラウドサーバ内の通知サーバからブラウザへのWebPush通知の確認,
クラウドサーバ内のWebアプリケーションによる行動履歴表示の確認
を行った.
結果として各機能の動作が正しく行われることが確認できた.
% また,無線モジュールからセンサデータの文字列をデータボディとして送信した場合,
% 「content-type:application/x-www-form-urlencoded」という,
% HTMLのフォームタグで送信する形式を用いて通信が行われると判明した.

\section{まとめと今後の課題}
クラウドサーバの機能である
畑から送られてくるセンサデータの受け取り,
農業従事者へのPush通知と行動履歴表示
について設計・実装および検証実験を行い,動作を確認できた.

しかしモジュールごとの実験は行ったが,
まだシステム全体を連携させて
実際の畑で得られたセンサーデータで解析,通知・表示を行うことができていないため,
行うことで実際の効果をさらに高めることが期待できる.
% 実際の効果を確認するには行うことが今後の課題である.

% 本稿担当であるクラウドサーバについて,
% セキュリティやUI・UXの向上の他に,
% リバースプロキシとデータベースアクセス制御用の内部APIが通信量のボトルネックとなる懸念点が挙げられる.
% 通信量のボトルネックは,本論文のシステム構成ではリバースプロキシと内部APIは
% データが集中するのに全体で一つしか用意していないことが根本的な原因である.
% そのため,kubernetesといった分散管理システムの導入することで,
% 更なる冗長性とスケーラビリティが実現することができる.

\bibliography{reference}
\bibliographystyle{junsrt}


\end{document}
