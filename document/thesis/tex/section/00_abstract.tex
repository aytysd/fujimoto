%! Tex root = ../main.tex
\expandafter\ifx\csname ifdraft\endcsname\relax
    \documentclass[12pt]{honka_v1}
    \usepackage{soturon}
    \begin{document}
\fi
%----- アブストラクト(概要)
% ここには研究論文の概要を書く.
% 研究としては,背景,目的,方法,結果が書かれているはずなので,
% それを1pにまとめて記述する.最後は研究結果なので,過去形で終わるはずである.
\begin{abstract}
% 背景
自然を相手にする農業従事者が抱える問題として獣害がある.
% 獣害被害の中でも夜間に発生するものは,夜間であるが故に視界が悪いことや,農業従事者が自分の身で直接の対策が困難である.
% これまでの対策としては,荒らされた畑や食べられた農作物の痕跡から動物を判断し,
% 電気柵やネットなどの設置という,動物やその行動に合った対策を対処的に行われている.
% しかし,柵に隙間がある場合や,そもそも想定していた動物でない,対処的な対策故に
% 原因を特定し適切な対策を取れるまでに大きな手間と時間・費用を要している.
獣害被害の中でも夜間に発生するものは,
害獣の侵入に気づくことが困難であり,日中と異なり農業従事者が直接追い払うことが難しい.
そのため基本的には獣害被害は事後に痕跡から経路や動物の種類を推定して電気柵や対策が取られている.
しかし,害獣の正確な侵入経路がわからないため,
柵の隙間といった不備のある位置がわからないことやそもそも推定した動物の種類が間違っているなど対策が機能しないことも少なくない.
そのため適切な対策を取るために様々な対策を試す必要があり,大きな手間と時間・費用を要している.
% 目的
そこで我々はこれらの問題を解決するために,
畑への侵入を検出・通知し,その動物の行動や大きさを推定することで,
農業従事者の獣害対策を効果的に行えるよう支援するシステムを提案する.
特に本論文ではクラウドサーバを用いて,
畑への侵入したという情報および推定して得られた結果を保存するデータベース,
畑への侵入情報から農業従事者へ通知するWebアプリケーション,
推定された行動や大きさを地図を用いて行動履歴を表示するWebアプリケーション
の設計を行った.
% 方法
機能としては,
リバースプロキシ,
WebPushを用いた通知サーバ,
Leaflet.jsによる地図を用いた行動履歴表示を行うWebアプリケーション,
データベースおよびデータベースアクセス制御用の内部API
があり,
それぞれをDocker Composeを駆使して連携させたクラウドサーバを設計した.
% 結果
設計・実装後,
クラウドサーバがやり取りする畑のゲートウェイに搭載される
無線モジュールとの疎通確認および通信形式の確認,
クラウドサーバ内の通知サーバからブラウザへのPush通知の確認,
クラウドサーバ内のWebアプリケーションによる行動履歴表示の確認
を行った.
さらなる実際の使用環境に合わせた改善をするには,
全体を連携させて実際の畑で得られたセンサーデータで解析,通知・表示を行うことが必要である.
% 展望
展望としては,本論文担当であるクラウドサーバについて
セキュリティやUI・UXの向上の他に
リバースプロキシとデータベースアクセス制御用の内部APIが通信量のボトルネックになるといった懸念点が挙げられる.
通信量のボトルネックは,本論文のシステム構成ではリバースプロキシと内部APIに
データが集中するのに対し1つずつしか用意していないことが根本的な原因である.
そのため,kubernetesといった分散管理システムの導入することで,
更なる冗長性とスケーラビリティが実現することができる.
\end{abstract}

% ソースコードの記述
% \mylisting{ファイルパス}{caption}{label}
% \mylisting{report.sty}{レポートのスタイルファイル}{fig:style}

% 図の配置
% \myfigure{ファイルパス}{caption}{label} サイズが画面横の10割
% \myfigureh{ファイルパス}{caption}{label} サイズが画面縦の10割
% \myfigurem{ファイルパス}{caption}{label} サイズが画面横の6割
% \myfigures{ファイルパス}{caption}{label}{size} サイズが画面横の{size}割

% \myfigure{figure.png}{画像ファイル}{fig:image}
% \reff{fig:image}

% 式の作成
% \begin{align}
%     \label{for:opamp_inout_char}
%     V_o = A \times (V_+ - V_-)
% \end{align}
% \refe{for:opamp_inout_char}

% 表の作成
% \reft{tb:}
% \begin{table}[htbp]
%     \caption{}
%     \label{tb:}
%     \centering
%     \begin{tabular}{|c|c|c|c|c|c|}\hline
%         器具名 & 個数 & メーカー & 型番 & シリアル番号 & 管理番号 \\ \hline
%         \hline
%         TTL-IC用直流電源 & 1台 & --- & --- & --- & 2班 \\ \hline
%     \end{tabular}
% \end{table}

% 箇条書き
% \begin{itemize}
%     \item 
% \end{itemize}

\expandafter\ifx\csname ifdraft\endcsname\relax
    % ----- 仕掛け 開始 -----
    \newcommand{\ifdraft}{false}
    % ----- 仕掛け 終了 -----
    %! Tex root = ../main.tex
\expandafter\ifx\csname ifdraft\endcsname\relax
    \documentclass[12pt]{honka_v1}
    \usepackage{soturon}
    \begin{document}
\fi
% 以下本文
% \printbibliography[title=参考文献]
% 参考文献をchapterとして目次に追加
% \addcontentsline{toc}{chapter}{参考文献}
\begin{thebibliography}{99}
    \bibitem{bib:jugaihigai-web} 農林水産省, ``野生鳥獣による農作物被害状況の推移'', \url{https://www.maff.go.jp/j/seisan/tyozyu/higai/hogai_zyoukyou/attach/pdf/index-31.pdf}, \refdate{2024年1月18日}.
    
    \bibitem{bib:docker-book} 大澤文孝, 浅井尚, ``触って学ぶクラウドインフラ docker 基礎からのコンテナ構築'', 日経BPマーケティング, 2020年.
    
    \bibitem{bib:node-book} 掌田津耶乃, ``Node.js 超入門'', 株式会社 秀和システム, 2017年.
    \bibitem{bib:express-web} StrongLoop, IBM, ``Express - Node.js web application framework'', \url{https://expressjs.com/}, \refdate{2024年1月18日}.
    \bibitem{bib:leafletjs-web} Volodymyr Agafonkin, ``Leaflet - a JavaScript library for interactive maps'', \url{https://leafletjs.com}, \refdate{2024年1月18日}.
    
    \bibitem{bib:pushapi-web} Mozilla Foundation, ``プッシュ API - Web API \textbar\ MDN'', \url{https://developer.mozilla.org/ja/docs/Web/API/Push_API}, \refdate{2024年1月18日}.
    \bibitem{bib:webpush-rfc} M. Thomson, E. Damaggio, B. Raymor, Ed., ``Generic Event Delivery Using HTTP Push'', \url{https://datatracker.ietf.org/doc/html/rfc8030}, RFC8030, December 2016, \refdate{2024年1月21日}.
    \bibitem{bib:serviceworker-web} Mozilla Foundation, ``サービスワーカー API - Web API \textbar\ MDN'', \url{https://developer.mozilla.org/ja/docs/Web/API/Service_Worker_API}, \refdate{2024年1月18日}.
    \bibitem{bib:vapid-rfc} M. Thomson, P. Beverloo, ``Voluntary Application Server Identification (VAPID)'', \url{https://datatracker.ietf.org/doc/html/rfc8292}, RFC8292, November 2017, \refdate{2024年1月21日}.
    
    \bibitem{bib:nginx-web} Nginx, ``nginx'', \url{https://nginx.org/en/}, \refdate{2024年1月18日}.
    \bibitem{bib:postform-web} Mozilla Foundation, ``POST - HTTP \textbar\ MDN'', \url{https://developer.mozilla.org/ja/docs/Web/HTTP/Methods/POST}, \refdate{2024年1月22日}.
    % \bibitem{bib:} , "", \url{https://developer.mozilla.org/ja/docs/Web/HTTP/Methods/POST}, \refdate{2024年1月22日}.
    
    
    % \bibitem{bib:Ishizaka}
    % 石坂守,山口賢一,岩田大志:``スキャン非同期記憶素子およびそれを備えた半導体集積回路ならびにその設計方法およびテストパターン'', 電子情報通信学会論文A,Vol.j102-A,No.6,pp.172-181,Jun.(2019)

    % \bibitem{textbook} \UTF{9AD9}玉 圭樹, "マルエージェント学習 -- 相互作用の謎に迫る --", 株式会社コロナ社, 2003年4月23日.
    % \bibitem{book} author, "title", publish, 2020.
    % \bibitem{webpage} author, "title", \url{URL}, \refdate{2021年8月15日}.
\end{thebibliography}

% なるべく書籍がいい
% 念の為の公式のテキストもメモ
% 
% * docker
% 本がある
% * docker-compose
% 本がある
% * Node.js
% * Leaflet.js
% https://leafletjs.com
% * PushAPI
% https://www.w3.org/TR/push-api/
% https://www.w3.org/standards/types/#WD
% https://developer.mozilla.org/en-US/docs/Web/API/Push_API
% * Nginx
% * Reverse Proxy

\expandafter\ifx\csname ifdraft\endcsname\relax
    % ----- 仕掛け 開始 -----
    \newcommand{\ifdraft}{false}
    % ----- 仕掛け 終了 -----
    % \include{references.tex}
    \end{document}
\fi
    \end{document}
\fi