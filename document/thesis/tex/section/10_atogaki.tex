%! Tex root = ../main.tex
\expandafter\ifx\csname ifdraft\endcsname\relax
    \documentclass[12pt]{honka_v1}
    \usepackage{soturon}
    \begin{document}
\fi
% 以下本文
\section{あとがき}
% \section{展望}
% 卒業研究の成果を総括して述べ,研究目標に対してどこまで到達できたかを吟味,検討し,
% その研究で成し得なかったことや研究の将来性に関することを記述する.

本論文では害獣検出と行動解析を行うシステムの中で
クラウドサーバに関する部分の設計・実装・動作確認について述べてきた.
畑に設置予定のゲートウェイとのデータ通信や
通知アプリの動作検証,
Webアプリケーションによる行動履歴の表示といった,
クラウドサーバとして必要な要素についての設計・検証を行ってきたが,
まだ試験的な部分が大きく,実際に利用するには至らない部分が多い.
特に優先的に対処しなければならないこととして以下に示すことが挙げられる.
\begin{itemize}
    \item セキュリティ対策
    \item UI・UXの向上
    \item サーバの冗長性
\end{itemize}

\subsection{セキュリティ対策}
クラウドサーバではゲートウェイからの受信にHTTP通信を用いている.
そのため,データ通信が十分に暗号化されておらず,
通信中に盗み見られることや悪意を持った攻撃者がデータを装って
POST送信をしてくることが考えられる.
これに関しては
ゲートウェイに搭載できるモジュールやマイコンの性能に左右されるため,
ゲートウェイ設計担当である共同研究者とも話し合っていく必要があると考えられる.

また,クラウドサーバへのアクセスは基本的に
クラウドサーバ自体のファイアウォールやDocker Compose内のリバースプロキシ,
利用するクラウドサービス独自のセキュリティシステムを考えている.
本論文のシステムではクラウドサーバ内で
農業従事者の情報を入力し,登録するフォームを設けるなど
それらの個人情報を取り扱うこととなるため,十分に堅牢なシステムが必要だと考えられる.
そのためデータ管理や通信手段をHTTPSなどで暗号化するなど,
登録方法を別の手段を利用して本クラウドサーバ上で個人情報のやり取りを
最小限にするなどの対策することが重要である.

\subsection{UI・UXの向上}
本論文のクラウドサーバは実験結果の通り,
機能を確かめるための仮組みの設計でしかない.
そのため対象ユーザが農業従事者であり,
基本的にパソコンなどの操作に慣れていない可能性を考慮する必要がある.


\subsection{サーバの冗長性}
本クラウドサーバは常時,畑に設定したゲートウェイから
センサ情報を受け取ることが予想される.
そのため,登録される畑が増えれば
クラウドサーバへのデータ送信は肥大化していく.
特に対象としているのが夜間の獣害被害であるため,
そのデータ送信も夜間に集中する可能性がある.
そうなったときに,本論文のクラウドサーバ構成は
通信制御を行う
リバースプロキシとデータベースがボトルネックとなることが考えられる.

そのような場合に備えて将来的にはボトルネックとなる,
Dockerコンテナを複数個並列して用意することや負荷を分散させる機構(kubernetesのようなロードバランサー)
を用いることなどが必要になると考えられる.



% セキュリティやUI・UXへの考慮


% 成果の総括

% 到達度について

% 将来性





% ソースコードの記述
% \mylisting{ファイルパス}{caption}{label}
% \mylisting{report.sty}{レポートのスタイルファイル}{fig:style}

% 図の配置
% \myfigure{ファイルパス}{caption}{label} サイズが画面横の10割
% \myfigureh{ファイルパス}{caption}{label} サイズが画面縦の10割
% \myfigurem{ファイルパス}{caption}{label} サイズが画面横の6割
% \myfigures{ファイルパス}{caption}{label}{size} サイズが画面横の{size}割

% \myfigure{figure.png}{画像ファイル}{fig:image}
% \reff{fig:image}

% 式の作成
% \begin{align}
%     \label{for:opamp_inout_char}
%     V_o = A \times (V_+ - V_-)
% \end{align}
% \refe{for:opamp_inout_char}

% 表の作成
% \reft{tb:}
% \begin{table}[htbp]
%     \caption{}
%     \label{tb:}
%     \centering
%     \begin{tabular}{|c|c|c|c|c|c|}\hline
%         器具名 & 個数 & メーカー & 型番 & シリアル番号 & 管理番号 \\ \hline
%         \hline
%         TTL-IC用直流電源 & 1台 & --- & --- & --- & 2班 \\ \hline
%     \end{tabular}
% \end{table}

% 箇条書き
% \begin{itemize}
%     \item 
% \end{itemize}

\expandafter\ifx\csname ifdraft\endcsname\relax
    % ----- 仕掛け 開始 -----
    \newcommand{\ifdraft}{false}
    % ----- 仕掛け 終了 -----
    %! Tex root = ../main.tex
\expandafter\ifx\csname ifdraft\endcsname\relax
    \documentclass[12pt]{honka_v1}
    \usepackage{soturon}
    \begin{document}
\fi
% 以下本文
% \printbibliography[title=参考文献]
% 参考文献をchapterとして目次に追加
% \addcontentsline{toc}{chapter}{参考文献}
\begin{thebibliography}{99}
    \bibitem{bib:jugaihigai-web} 農林水産省, ``野生鳥獣による農作物被害状況の推移'', \url{https://www.maff.go.jp/j/seisan/tyozyu/higai/hogai_zyoukyou/attach/pdf/index-31.pdf}, \refdate{2024年1月18日}.
    
    \bibitem{bib:docker-book} 大澤文孝, 浅井尚, ``触って学ぶクラウドインフラ docker 基礎からのコンテナ構築'', 日経BPマーケティング, 2020年.
    
    \bibitem{bib:node-book} 掌田津耶乃, ``Node.js 超入門'', 株式会社 秀和システム, 2017年.
    \bibitem{bib:express-web} StrongLoop, IBM, ``Express - Node.js web application framework'', \url{https://expressjs.com/}, \refdate{2024年1月18日}.
    \bibitem{bib:leafletjs-web} Volodymyr Agafonkin, ``Leaflet - a JavaScript library for interactive maps'', \url{https://leafletjs.com}, \refdate{2024年1月18日}.
    
    \bibitem{bib:pushapi-web} Mozilla Foundation, ``プッシュ API - Web API \textbar\ MDN'', \url{https://developer.mozilla.org/ja/docs/Web/API/Push_API}, \refdate{2024年1月18日}.
    \bibitem{bib:webpush-rfc} M. Thomson, E. Damaggio, B. Raymor, Ed., ``Generic Event Delivery Using HTTP Push'', \url{https://datatracker.ietf.org/doc/html/rfc8030}, RFC8030, December 2016, \refdate{2024年1月21日}.
    \bibitem{bib:serviceworker-web} Mozilla Foundation, ``サービスワーカー API - Web API \textbar\ MDN'', \url{https://developer.mozilla.org/ja/docs/Web/API/Service_Worker_API}, \refdate{2024年1月18日}.
    \bibitem{bib:vapid-rfc} M. Thomson, P. Beverloo, ``Voluntary Application Server Identification (VAPID)'', \url{https://datatracker.ietf.org/doc/html/rfc8292}, RFC8292, November 2017, \refdate{2024年1月21日}.
    
    \bibitem{bib:nginx-web} Nginx, ``nginx'', \url{https://nginx.org/en/}, \refdate{2024年1月18日}.
    \bibitem{bib:postform-web} Mozilla Foundation, ``POST - HTTP \textbar\ MDN'', \url{https://developer.mozilla.org/ja/docs/Web/HTTP/Methods/POST}, \refdate{2024年1月22日}.
    % \bibitem{bib:} , "", \url{https://developer.mozilla.org/ja/docs/Web/HTTP/Methods/POST}, \refdate{2024年1月22日}.
    
    
    % \bibitem{bib:Ishizaka}
    % 石坂守,山口賢一,岩田大志:``スキャン非同期記憶素子およびそれを備えた半導体集積回路ならびにその設計方法およびテストパターン'', 電子情報通信学会論文A,Vol.j102-A,No.6,pp.172-181,Jun.(2019)

    % \bibitem{textbook} \UTF{9AD9}玉 圭樹, "マルエージェント学習 -- 相互作用の謎に迫る --", 株式会社コロナ社, 2003年4月23日.
    % \bibitem{book} author, "title", publish, 2020.
    % \bibitem{webpage} author, "title", \url{URL}, \refdate{2021年8月15日}.
\end{thebibliography}

% なるべく書籍がいい
% 念の為の公式のテキストもメモ
% 
% * docker
% 本がある
% * docker-compose
% 本がある
% * Node.js
% * Leaflet.js
% https://leafletjs.com
% * PushAPI
% https://www.w3.org/TR/push-api/
% https://www.w3.org/standards/types/#WD
% https://developer.mozilla.org/en-US/docs/Web/API/Push_API
% * Nginx
% * Reverse Proxy

\expandafter\ifx\csname ifdraft\endcsname\relax
    % ----- 仕掛け 開始 -----
    \newcommand{\ifdraft}{false}
    % ----- 仕掛け 終了 -----
    % \include{references.tex}
    \end{document}
\fi
    \end{document}
\fi