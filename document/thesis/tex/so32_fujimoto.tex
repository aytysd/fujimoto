\documentclass[12pt]{honka_v1}

\usepackage{soturon}
\usepackage{parameter}

% ----- 仕掛け 開始 -----
\newcommand{\ifdraft}{false}
% ----- 仕掛け 終了 -----


\setcounter{secnumdepth}{4}


\begin{document} % 論文の始まり始まり~

% \titlepage
%----- アブストラクト(概要)
% ここには研究論文の概要を書く.
% 研究としては,背景,目的,方法,結果が書かれているはずなので,
% それを1pにまとめて記述する.最後は研究結果なので,過去形で終わるはずである.
% \begin{abstract}

% \end{abstract}
%! Tex root = ../main.tex
\expandafter\ifx\csname ifdraft\endcsname\relax
    \documentclass[12pt]{honka_v1}
    \usepackage{soturon}
    \begin{document}
\fi
%----- アブストラクト(概要)
% ここには研究論文の概要を書く.
% 研究としては,背景,目的,方法,結果が書かれているはずなので,
% それを1pにまとめて記述する.最後は研究結果なので,過去形で終わるはずである.
\begin{abstract}
% 背景
自然を相手にする農業従事者が抱える問題として獣害がある.
% 獣害被害の中でも夜間に発生するものは,夜間であるが故に視界が悪いことや,農業従事者が自分の身で直接の対策が困難である.
% これまでの対策としては,荒らされた畑や食べられた農作物の痕跡から動物を判断し,
% 電気柵やネットなどの設置という,動物やその行動に合った対策を対処的に行われている.
% しかし,柵に隙間がある場合や,そもそも想定していた動物でない,対処的な対策故に
% 原因を特定し適切な対策を取れるまでに大きな手間と時間・費用を要している.
獣害被害の中でも夜間に発生するものは,
害獣の侵入に気づくことが困難であり,日中と異なり農業従事者が直接追い払うことが難しい.
そのため基本的には獣害被害は事後に痕跡から経路や動物の種類を推定して電気柵や対策が取られている.
しかし,害獣の正確な侵入経路がわからないため,
柵の隙間といった不備のある位置がわからないことやそもそも推定した動物の種類が間違っているなど対策が機能しないことも少なくない.
そのため適切な対策を取るために様々な対策を試す必要があり,大きな手間と時間・費用を要している.
% 目的
そこで我々はこれらの問題を解決するために,
畑への侵入を検出・通知し,その動物の行動や大きさを推定することで,
農業従事者の獣害対策を効果的に行えるよう支援するシステムを提案する.
特に本論文ではクラウドサーバを用いて,
畑への侵入したという情報および推定して得られた結果を保存するデータベース,
畑への侵入情報から農業従事者へ通知するWebアプリケーション,
推定された行動や大きさを地図を用いて行動履歴を表示するWebアプリケーション
の設計を行った.
% 方法
機能としては,
リバースプロキシ,
WebPushを用いた通知サーバ,
Leaflet.jsによる地図を用いた行動履歴表示を行うWebアプリケーション,
データベースおよびデータベースアクセス制御用の内部API
があり,
それぞれをDocker Composeを駆使して連携させたクラウドサーバを設計した.
% 結果
設計・実装後,
クラウドサーバがやり取りする畑のゲートウェイに搭載される
無線モジュールとの疎通確認および通信形式の確認,
クラウドサーバ内の通知サーバからブラウザへのPush通知の確認,
クラウドサーバ内のWebアプリケーションによる行動履歴表示の確認
を行った.
さらなる実際の使用環境に合わせた改善をするには,
全体を連携させて実際の畑で得られたセンサーデータで解析,通知・表示を行うことが必要である.
% 展望
展望としては,本論文担当であるクラウドサーバについて
セキュリティやUI・UXの向上の他に
リバースプロキシとデータベースアクセス制御用の内部APIが通信量のボトルネックになるといった懸念点が挙げられる.
通信量のボトルネックは,本論文のシステム構成ではリバースプロキシと内部APIに
データが集中するのに対し1つずつしか用意していないことが根本的な原因である.
そのため,kubernetesといった分散管理システムの導入することで,
更なる冗長性とスケーラビリティが実現することができる.
\end{abstract}

% ソースコードの記述
% \mylisting{ファイルパス}{caption}{label}
% \mylisting{report.sty}{レポートのスタイルファイル}{fig:style}

% 図の配置
% \myfigure{ファイルパス}{caption}{label} サイズが画面横の10割
% \myfigureh{ファイルパス}{caption}{label} サイズが画面縦の10割
% \myfigurem{ファイルパス}{caption}{label} サイズが画面横の6割
% \myfigures{ファイルパス}{caption}{label}{size} サイズが画面横の{size}割

% \myfigure{figure.png}{画像ファイル}{fig:image}
% \reff{fig:image}

% 式の作成
% \begin{align}
%     \label{for:opamp_inout_char}
%     V_o = A \times (V_+ - V_-)
% \end{align}
% \refe{for:opamp_inout_char}

% 表の作成
% \reft{tb:}
% \begin{table}[htbp]
%     \caption{}
%     \label{tb:}
%     \centering
%     \begin{tabular}{|c|c|c|c|c|c|}\hline
%         器具名 & 個数 & メーカー & 型番 & シリアル番号 & 管理番号 \\ \hline
%         \hline
%         TTL-IC用直流電源 & 1台 & --- & --- & --- & 2班 \\ \hline
%     \end{tabular}
% \end{table}

% 箇条書き
% \begin{itemize}
%     \item 
% \end{itemize}

\expandafter\ifx\csname ifdraft\endcsname\relax
    % ----- 仕掛け 開始 -----
    \newcommand{\ifdraft}{false}
    % ----- 仕掛け 終了 -----
    %! Tex root = ../main.tex
\expandafter\ifx\csname ifdraft\endcsname\relax
    \documentclass[12pt]{honka_v1}
    \usepackage{soturon}
    \begin{document}
\fi
% 以下本文
% \printbibliography[title=参考文献]
% 参考文献をchapterとして目次に追加
% \addcontentsline{toc}{chapter}{参考文献}
\begin{thebibliography}{99}
    \bibitem{bib:jugaihigai-web} 農林水産省, ``野生鳥獣による農作物被害状況の推移'', \url{https://www.maff.go.jp/j/seisan/tyozyu/higai/hogai_zyoukyou/attach/pdf/index-31.pdf}, \refdate{2024年1月18日}.
    
    \bibitem{bib:docker-book} 大澤文孝, 浅井尚, ``触って学ぶクラウドインフラ docker 基礎からのコンテナ構築'', 日経BPマーケティング, 2020年.
    
    \bibitem{bib:node-book} 掌田津耶乃, ``Node.js 超入門'', 株式会社 秀和システム, 2017年.
    \bibitem{bib:express-web} StrongLoop, IBM, ``Express - Node.js web application framework'', \url{https://expressjs.com/}, \refdate{2024年1月18日}.
    \bibitem{bib:leafletjs-web} Volodymyr Agafonkin, ``Leaflet - a JavaScript library for interactive maps'', \url{https://leafletjs.com}, \refdate{2024年1月18日}.
    
    \bibitem{bib:pushapi-web} Mozilla Foundation, ``プッシュ API - Web API \textbar\ MDN'', \url{https://developer.mozilla.org/ja/docs/Web/API/Push_API}, \refdate{2024年1月18日}.
    \bibitem{bib:webpush-rfc} M. Thomson, E. Damaggio, B. Raymor, Ed., ``Generic Event Delivery Using HTTP Push'', \url{https://datatracker.ietf.org/doc/html/rfc8030}, RFC8030, December 2016, \refdate{2024年1月21日}.
    \bibitem{bib:serviceworker-web} Mozilla Foundation, ``サービスワーカー API - Web API \textbar\ MDN'', \url{https://developer.mozilla.org/ja/docs/Web/API/Service_Worker_API}, \refdate{2024年1月18日}.
    \bibitem{bib:vapid-rfc} M. Thomson, P. Beverloo, ``Voluntary Application Server Identification (VAPID)'', \url{https://datatracker.ietf.org/doc/html/rfc8292}, RFC8292, November 2017, \refdate{2024年1月21日}.
    
    \bibitem{bib:nginx-web} Nginx, ``nginx'', \url{https://nginx.org/en/}, \refdate{2024年1月18日}.
    \bibitem{bib:postform-web} Mozilla Foundation, ``POST - HTTP \textbar\ MDN'', \url{https://developer.mozilla.org/ja/docs/Web/HTTP/Methods/POST}, \refdate{2024年1月22日}.
    % \bibitem{bib:} , "", \url{https://developer.mozilla.org/ja/docs/Web/HTTP/Methods/POST}, \refdate{2024年1月22日}.
    
    
    % \bibitem{bib:Ishizaka}
    % 石坂守,山口賢一,岩田大志:``スキャン非同期記憶素子およびそれを備えた半導体集積回路ならびにその設計方法およびテストパターン'', 電子情報通信学会論文A,Vol.j102-A,No.6,pp.172-181,Jun.(2019)

    % \bibitem{textbook} \UTF{9AD9}玉 圭樹, "マルエージェント学習 -- 相互作用の謎に迫る --", 株式会社コロナ社, 2003年4月23日.
    % \bibitem{book} author, "title", publish, 2020.
    % \bibitem{webpage} author, "title", \url{URL}, \refdate{2021年8月15日}.
\end{thebibliography}

% なるべく書籍がいい
% 念の為の公式のテキストもメモ
% 
% * docker
% 本がある
% * docker-compose
% 本がある
% * Node.js
% * Leaflet.js
% https://leafletjs.com
% * PushAPI
% https://www.w3.org/TR/push-api/
% https://www.w3.org/standards/types/#WD
% https://developer.mozilla.org/en-US/docs/Web/API/Push_API
% * Nginx
% * Reverse Proxy

\expandafter\ifx\csname ifdraft\endcsname\relax
    % ----- 仕掛け 開始 -----
    \newcommand{\ifdraft}{false}
    % ----- 仕掛け 終了 -----
    % \include{references.tex}
    \end{document}
\fi
    \end{document}
\fi

\tableofcontents

%==========[ beginning of main Subject ]==========%
\start
%! Tex root = ../main.tex
\expandafter\ifx\csname ifdraft\endcsname\relax
    \documentclass[12pt]{honka_v1}
    \usepackage{soturon}
    \begin{document}
\fi
% 以下本文
\section{まえがき}
% ここでは,卒業研究のテーマに対する歴史的背景および研究の動機と目的について述べ,
% この研究分野における位置付けを明らかにし,
% 自分の研究が従来の研究とどのような関係にあるのか,
% 研究結果がどのような価値を持つのかを明確に述べておく.
自然を相手にする農業従事者が抱える問題として獣害がある.
令和4年度の野生鳥獣による全国の農作物被害は約156億円で,
特にシカの被害額は約65億円で前年の被害額と比べて増加傾向にある\cite{bib:jugaihigai-web}.

% 獣害被害の中でも夜間に発生するものは,
% 夜間であるが故に視界が悪いことや,農業従事者が自分の身で直接の対策が困難である.
% これまでの対策としては,荒らされた畑や食べられた農作物の痕跡から動物を判断し,
% 電気柵やネットなどの設置という,動物やその行動に合った対策を対処的に行われている.
% しかし,柵に隙間があったり,そもそも想定していた動物でない,対処的な対策故に
% 原因を特定し適切な対策を取れるまでに大きな手間と時間・費用を要している.
獣害被害の中でも夜間に発生するものは,
害獣の侵入に気づくことが困難であり,日中と異なり農業従事者が直接追い払うことが難しい.
そのため基本的には獣害被害は事後に痕跡から経路や動物の種類を推定して電気柵や対策が取られている.
しかし,害獣の正確な侵入経路がわからないため
柵の隙間といった不備のある位置がわからないことやそもそも推定した動物の種類が間違っているなど対策が機能しないことも少なくない.
そのため適切な対策を取るために様々な対策を試す必要があり,大きな手間と時間・費用を要している.


我々はこれらの問題を解決するために,
畑への侵入を検出・通知し,その動物の行動や大きさを推定することで
農業従事者の獣害対策を効果的に行えるよう支援するシステムを提案する.
具体的な流れとしては,畑の周囲に赤外線センサを搭載したセンサノードを規則的に配置する.
赤外線センサノードが検出した侵入した動物のデータをLoRa通信およびLTE通信を用いてクラウドサーバへ送信する.
クラウドサーバでは収集されたセンサデータを用いて農業従事者に通知や
センサデータの解析で得られた畑内に侵入した動物についての行動履歴を地図と合わせて農業従事者に提示する.

本論文では著者が担当するクラウドサーバの解析機能以外の機能群について述べる.
クラウドサーバの著者担当部分は大きく分けて,データの保存,農業従事者への通知,行動履歴の表示の
3つがある.
データの保存では
畑に設置した赤外線センサノードから得られたセンサデータ,
センサデータから解析して得られた行動履歴のデータ,
登録されている畑についてのデータ
について保存し,管理する.
農業従事者への通知ではセンサデータの送信に応答して農業従事者の個人端末へ通知を行う
Webアプリケーションを実現する.
行動履歴の表示ではクラウドサーバ内に保存された行動履歴のデータを用いて
農業従事者の個人端末で確認可能なWebアプリケーションを実現する.
以上の3つの機能について
設計・実装をし,動作確認を行う.

% クラウドサーバでは赤外線センサノードから得られたデータを保存し,
% データの解析結果は履歴として保存しつつ,農業従事者がフィードバックとして確認できるように,
% 地図上に解析による行動などの結果を表示するWebアプリケーションを提供する.
% また,農業従事者によるリアルタイムの対処を実現するために,
% クラウドサーバでは赤外線センサノードからのデータを受け取ると農業従事者に即時通知を行う.


% 特に本論文ではクラウドサーバが担うセンサノードからのデータ受け取り・管理,
% データ解析結果のビジュアライザWebアプリケーションケーション,
% 農業従事者への即時通知機能の設計および実装し,動作確認を行う.

% ソースコードの記述
% \mylisting{ファイルパス}{caption}{label}
% \mylisting{report.sty}{レポートのスタイルファイル}{fig:style}

% 図の配置
% \myfigure{ファイルパス}{caption}{label} サイズが画面横の10割
% \myfigureh{ファイルパス}{caption}{label} サイズが画面縦の10割
% \myfigurem{ファイルパス}{caption}{label} サイズが画面横の6割
% \myfigures{ファイルパス}{caption}{label}{size} サイズが画面横の{size}割

% \myfigure{figure.png}{画像ファイル}{fig:image}
% \reff{fig:image}

% 式の作成
% \begin{align}
%     \label{for:opamp_inout_char}
%     V_o = A \times (V_+ - V_-)
% \end{align}
% \refe{for:opamp_inout_char}

% 表の作成
% \reft{tb:}
% \begin{table}[htbp]
%     \caption{}
%     \label{tb:}
%     \centering
%     \begin{tabular}{|c|c|c|c|c|c|}\hline
%         器具名 & 個数 & メーカー & 型番 & シリアル番号 & 管理番号 \\ \hline
%         \hline
%         TTL-IC用直流電源 & 1台 & --- & --- & --- & 2班 \\ \hline
%     \end{tabular}
% \end{table}

% 箇条書き
% \begin{itemize}
%     \item 
% \end{itemize}

\expandafter\ifx\csname ifdraft\endcsname\relax
    % ----- 仕掛け 開始 -----
    \newcommand{\ifdraft}{false}
    % ----- 仕掛け 終了 -----
    %! Tex root = ../main.tex
\expandafter\ifx\csname ifdraft\endcsname\relax
    \documentclass[12pt]{honka_v1}
    \usepackage{soturon}
    \begin{document}
\fi
% 以下本文
% \printbibliography[title=参考文献]
% 参考文献をchapterとして目次に追加
% \addcontentsline{toc}{chapter}{参考文献}
\begin{thebibliography}{99}
    \bibitem{bib:jugaihigai-web} 農林水産省, ``野生鳥獣による農作物被害状況の推移'', \url{https://www.maff.go.jp/j/seisan/tyozyu/higai/hogai_zyoukyou/attach/pdf/index-31.pdf}, \refdate{2024年1月18日}.
    
    \bibitem{bib:docker-book} 大澤文孝, 浅井尚, ``触って学ぶクラウドインフラ docker 基礎からのコンテナ構築'', 日経BPマーケティング, 2020年.
    
    \bibitem{bib:node-book} 掌田津耶乃, ``Node.js 超入門'', 株式会社 秀和システム, 2017年.
    \bibitem{bib:express-web} StrongLoop, IBM, ``Express - Node.js web application framework'', \url{https://expressjs.com/}, \refdate{2024年1月18日}.
    \bibitem{bib:leafletjs-web} Volodymyr Agafonkin, ``Leaflet - a JavaScript library for interactive maps'', \url{https://leafletjs.com}, \refdate{2024年1月18日}.
    
    \bibitem{bib:pushapi-web} Mozilla Foundation, ``プッシュ API - Web API \textbar\ MDN'', \url{https://developer.mozilla.org/ja/docs/Web/API/Push_API}, \refdate{2024年1月18日}.
    \bibitem{bib:webpush-rfc} M. Thomson, E. Damaggio, B. Raymor, Ed., ``Generic Event Delivery Using HTTP Push'', \url{https://datatracker.ietf.org/doc/html/rfc8030}, RFC8030, December 2016, \refdate{2024年1月21日}.
    \bibitem{bib:serviceworker-web} Mozilla Foundation, ``サービスワーカー API - Web API \textbar\ MDN'', \url{https://developer.mozilla.org/ja/docs/Web/API/Service_Worker_API}, \refdate{2024年1月18日}.
    \bibitem{bib:vapid-rfc} M. Thomson, P. Beverloo, ``Voluntary Application Server Identification (VAPID)'', \url{https://datatracker.ietf.org/doc/html/rfc8292}, RFC8292, November 2017, \refdate{2024年1月21日}.
    
    \bibitem{bib:nginx-web} Nginx, ``nginx'', \url{https://nginx.org/en/}, \refdate{2024年1月18日}.
    \bibitem{bib:postform-web} Mozilla Foundation, ``POST - HTTP \textbar\ MDN'', \url{https://developer.mozilla.org/ja/docs/Web/HTTP/Methods/POST}, \refdate{2024年1月22日}.
    % \bibitem{bib:} , "", \url{https://developer.mozilla.org/ja/docs/Web/HTTP/Methods/POST}, \refdate{2024年1月22日}.
    
    
    % \bibitem{bib:Ishizaka}
    % 石坂守,山口賢一,岩田大志:``スキャン非同期記憶素子およびそれを備えた半導体集積回路ならびにその設計方法およびテストパターン'', 電子情報通信学会論文A,Vol.j102-A,No.6,pp.172-181,Jun.(2019)

    % \bibitem{textbook} \UTF{9AD9}玉 圭樹, "マルエージェント学習 -- 相互作用の謎に迫る --", 株式会社コロナ社, 2003年4月23日.
    % \bibitem{book} author, "title", publish, 2020.
    % \bibitem{webpage} author, "title", \url{URL}, \refdate{2021年8月15日}.
\end{thebibliography}

% なるべく書籍がいい
% 念の為の公式のテキストもメモ
% 
% * docker
% 本がある
% * docker-compose
% 本がある
% * Node.js
% * Leaflet.js
% https://leafletjs.com
% * PushAPI
% https://www.w3.org/TR/push-api/
% https://www.w3.org/standards/types/#WD
% https://developer.mozilla.org/en-US/docs/Web/API/Push_API
% * Nginx
% * Reverse Proxy

\expandafter\ifx\csname ifdraft\endcsname\relax
    % ----- 仕掛け 開始 -----
    \newcommand{\ifdraft}{false}
    % ----- 仕掛け 終了 -----
    % \include{references.tex}
    \end{document}
\fi
    \end{document}
\fi

%! Tex root = ../main.tex
\expandafter\ifx\csname ifdraft\endcsname\relax
    \documentclass[12pt]{honka_v1}
    \usepackage{soturon}
    \begin{document}
\fi
% 以下本文
\section{理論}
本章では,本研究の前提となる知見や技術について説明する.
\subsection{Docker}
Dockerはコンテナ技術の一種で,
システムの環境ごとに隔離した空間を用意して独自のディレクトリツリーを提供する.
Dockerコンテナはそれ単体でシステムが完結しているため,
Dockerコンテナを別のサーバにコピーして起動させることも容易であり,ポータビリティ性を有している\cite{bib:docker-book}.
本論文ではDockerを用いて機能ごとにDockerコンテナを作成する.

\subsubsection{Dockerイメージ}
DockerイメージはDockerコンテナ作成を支援するために用意されたアーカイブパッケージである.
Dockerコンテナは独立したシステム実行環境であるため,
一から構築するにはOSやライブラリなど全てを作り込む必要があり非常に手間がかかる.
Dockerコンテナ作成に必要なファイルを纏めたアーカイブパッケージがDockerイメージである\cite{bib:docker-book}.
DockerイメージはDocker社が運営しているDocker Hubで登録・公開されており,
Docker Hubに公開されているDockerイメージをダウンロードしてDockerコンテナを作成することが可能である.

Dockerイメージには以下の2種類ある.
\begin{itemize}
    \item 基本的なLinuxディストリビューションだけのDockerイメージ
    \item アプリケーション入りのDockerイメージ
\end{itemize}

本論文では後者のアプリケーション入りのDockerイメージを用いてシステムのDockerコンテナを作成する.


\subsubsection{Dockerfile}
Dockerfileは自作のイメージを作るもので,
ベースとなるイメージに対してどのような変更指示をするかをまとめたファイルである.
Dockerfileは「docker build」コマンドを用いてビルドすることでDockerイメージを作成できる\cite{bib:docker-book}.

\subsubsection{Dockerネットワーク}
Dockerコンテナは独立しており,1つのDockerコンテナがサーバマシンのような役割を持っている.
Dockerコンテナ間が通信するためにDocker内に仮想的なネットワークを構築する必要がある.
Dockerには以下の3種類のネットワークが用意されている.
\begin{itemize}
    \item bridgeネットワーク
    \item hostネットワーク
    \item noneネットワーク
\end{itemize}
ここでは,本論文で利用するbridgeのみ説明し,他2種類については説明を省く.

bridgeネットワークではIPマスカレードを利用してホストPCが所属する新たな内部ネットワークを構築している.
\reff{fig:bridge-net}にApacheのDockerコンテナを2つ起動し,
それぞれのポートをホストPCの8080と8081に接続した時の例を示す.

\myfigurem{./images/bridgenetwork.pdf}{bridgeネットワーク例}{fig:bridge-net}

\subsubsection{Docker Compose}
Docker Composeは複数のDockerコンテナの作成およびDockerコンテナ間のネットワークの作成など,
複数のDockerコンテナの作成・停止・破棄の一連の操作をまとめて実行する仕組みのことである\cite{bib:docker-book}.
一連のDockerコンテナやシステムについての定義ファイル(Composeファイル)は既定では「docker-compose.yml」というファイル名になっている.
Composeファイルを「docker-compose(docker compose)」コマンドで実行すると
ボリュームやネットワークが作成され,まとめてDockerコンテナを起動できる.
停止や削除するときも同様のコマンドで行うことができる.
本論文で実装する通知や行動履歴表示といった一連のシステム
をまとめて起動・制御するためにDocker Composeを用いる.

\subsection{Node.js}
Node.jsはJavaScriptのランタイム環境であり,
Webブラウザの中でなく単独でJavaScriptのプログラムを実行できるようになっているものである\cite{bib:node-book}.
ランタイム環境であることで,従来のJavaScriptではクライアント側の実装でしか使えなかったが,
サーバ側でもJavaScriptを利用でき,Web開発の全てをJavaScriptのみで完結することができるようになっている.
Node.jsはWebサーバのための機能を備えており,容易にサーバ側の実装を行うことができる.

\subsubsection{npm}
npmはNode.js専用のパッケージマネージャの一種である\cite{bib:node-book}.
Node.jsではNode.jsの機能を拡張する様々なプログラムがパッケージとして配布されている.
その多種多様なパッケージをインストール・管理するシステムとしてnpmが用意されている.

\subsubsection{Express}
ExpressはNode.jsのフレームワークの一種で,
Node.jsの機能をわかりやすくし基本的なWebアプリケーション機能を比較的軽量で提供できる\cite{bib:node-book}.
% \subsubsection{routing}

\subsection{Leaflet.js}
Leaflet.jsはVolodymyr Agafonkin氏によって作成されたJavaScriptのマップライブラリである.
マーカの表示やポップアップ,地図の拡大縮小など基本的な地図アプリの機能を提供しており,
EdgeやChrome,Firefox,Safariなどの基本的なブラウザをサポートしている\cite{bib:leafletjs-web}.

本論文では畑の地図表示および害獣の行動経路を線として,センサノードの位置をマーカとして図示することに用いる.

\subsection{WebPush}
WebPushはWebアプリケーションがサーバからメッセージを受信できる仕組みのことである.
Webアプリケーションがフォアグランド状態や読み込まれているかどうかに関わらず,
リアルタイムな通知を提供することができる.
WebアプリケーションがPush通知のメッセージを送信するには受信側が「サービスワーカー」を登録している必要がある\cite{bib:pushapi-web}.

WebPushの一般的なモデルを\reff{fig:webpush-overview}に示す.
WebPushはまず利用者がPushサーバに登録を行い,同時にアプリケーションサーバに登録情報を共有する.
この時にサービスワーカーを利用者側に登録する.
アプリケーションサーバは共有された登録情報を用いてPushメッセージを
Pushサーバを経由して利用者に送り,Push通知を実現する\cite{bib:webpush-rfc}.

\myfigure{./images/webpush-overview.pdf}{WebPushの模式図}{fig:webpush-overview}

WebPushは基本的なブラウザで保証されていて,\reft{tb:WebPush-browser}に示すブラウザでは保証が確認されている\cite{bib:pushapi-web}.

\begin{table}[htbp]
    \caption{WebPushが保証されているブラウザ\cite{bib:pushapi-web}}
    \label{tb:WebPush-browser}
    \centering
    \begin{tabular}{c|c}\hline
        パソコン & スマートフォン \\
        \hline
        Chrome & Chrome Android\\
        Firefox & Firefox for Android\\
        Opera & Opera Android \\
        Safari & Safari on iOS\\
        Edge & Samsung Internet\\
        \hline
    \end{tabular}
\end{table}
% \begin{itemize}
%     \item Chrome
%     \item Edge
%     \item Firefox
%     \item Opera
%     \item Safari
%     \item Chrome Android
%     \item Firefox for Android
%     \item Opera Android
%     \item Safari on iOS
%     \item Samsung Internet
% \end{itemize}

\subsubsection{サービスワーカー(Service Worker)}
サービスワーカーはあるオリジン(プロトコル・ポート番号・ホストの3つの組)とパスに対して登録された
イベント駆動型のJavaScriptファイルである\cite{bib:serviceworker-web}.

\subsubsection{VAPID(Voluntary Application Server Identification)}
VAPIDはアプリケーションサーバをPushサービスに認証・識別してもらうための仕組みである\cite{bib:vapid-rfc}.
RFC8030\cite{bib:webpush-rfc}に記載されているWebPushにはアプリケーションサーバを認証する構造が含まれておらず,
アプリケーションサーバになりすました通信が行われる問題点がある.
ここでVAPIDによってアプリケーションサーバの識別をすることで,
アプリケーションサーバになりすました通信を防ぐことができるようになる.


\subsection{Nginx}
Nginxは元々Igor Sysoev氏が作成したHTTPサーバやリバースプロキシサーバ,メールプロキシサーバといった
汎用的なTCP/UDPプロキシサーバである\cite{bib:nginx-web}.
本論文においてはクラウドサーバ内のリバースプロキシサーバとして活用する.

% \subsection{Reverse Proxy}


% ソースコードの記述
% \mylisting{ファイルパス}{caption}{label}
% \mylisting{report.sty}{レポートのスタイルファイル}{fig:style}

% 図の配置
% \myfigure{ファイルパス}{caption}{label} サイズが画面横の10割
% \myfigureh{ファイルパス}{caption}{label} サイズが画面縦の10割
% \myfigurem{ファイルパス}{caption}{label} サイズが画面横の6割
% \myfigures{ファイルパス}{caption}{label}{size} サイズが画面横の{size}割

% \myfigure{figure.png}{画像ファイル}{fig:image}
% \reff{fig:image}

% 式の作成
% \begin{align}
%     \label{for:opamp_inout_char}
%     V_o = A \times (V_+ - V_-)
% \end{align}
% \refe{for:opamp_inout_char}

% 表の作成
% \reft{tb:}
% \begin{table}[htbp]
%     \caption{}
%     \label{tb:}
%     \centering
%     \begin{tabular}{|c|c|c|c|c|c|}\hline
%         器具名 & 個数 & メーカー & 型番 & シリアル番号 & 管理番号 \\ \hline
%         \hline
%         TTL-IC用直流電源 & 1台 & --- & --- & --- & 2班 \\ \hline
%     \end{tabular}
% \end{table}

% 箇条書き
% \begin{itemize}
%     \item 
% \end{itemize}

\expandafter\ifx\csname ifdraft\endcsname\relax
    % ----- 仕掛け 開始 -----
    \newcommand{\ifdraft}{false}
    % ----- 仕掛け 終了 -----
    %! Tex root = ../main.tex
\expandafter\ifx\csname ifdraft\endcsname\relax
    \documentclass[12pt]{honka_v1}
    \usepackage{soturon}
    \begin{document}
\fi
% 以下本文
% \printbibliography[title=参考文献]
% 参考文献をchapterとして目次に追加
% \addcontentsline{toc}{chapter}{参考文献}
\begin{thebibliography}{99}
    \bibitem{bib:jugaihigai-web} 農林水産省, ``野生鳥獣による農作物被害状況の推移'', \url{https://www.maff.go.jp/j/seisan/tyozyu/higai/hogai_zyoukyou/attach/pdf/index-31.pdf}, \refdate{2024年1月18日}.
    
    \bibitem{bib:docker-book} 大澤文孝, 浅井尚, ``触って学ぶクラウドインフラ docker 基礎からのコンテナ構築'', 日経BPマーケティング, 2020年.
    
    \bibitem{bib:node-book} 掌田津耶乃, ``Node.js 超入門'', 株式会社 秀和システム, 2017年.
    \bibitem{bib:express-web} StrongLoop, IBM, ``Express - Node.js web application framework'', \url{https://expressjs.com/}, \refdate{2024年1月18日}.
    \bibitem{bib:leafletjs-web} Volodymyr Agafonkin, ``Leaflet - a JavaScript library for interactive maps'', \url{https://leafletjs.com}, \refdate{2024年1月18日}.
    
    \bibitem{bib:pushapi-web} Mozilla Foundation, ``プッシュ API - Web API \textbar\ MDN'', \url{https://developer.mozilla.org/ja/docs/Web/API/Push_API}, \refdate{2024年1月18日}.
    \bibitem{bib:webpush-rfc} M. Thomson, E. Damaggio, B. Raymor, Ed., ``Generic Event Delivery Using HTTP Push'', \url{https://datatracker.ietf.org/doc/html/rfc8030}, RFC8030, December 2016, \refdate{2024年1月21日}.
    \bibitem{bib:serviceworker-web} Mozilla Foundation, ``サービスワーカー API - Web API \textbar\ MDN'', \url{https://developer.mozilla.org/ja/docs/Web/API/Service_Worker_API}, \refdate{2024年1月18日}.
    \bibitem{bib:vapid-rfc} M. Thomson, P. Beverloo, ``Voluntary Application Server Identification (VAPID)'', \url{https://datatracker.ietf.org/doc/html/rfc8292}, RFC8292, November 2017, \refdate{2024年1月21日}.
    
    \bibitem{bib:nginx-web} Nginx, ``nginx'', \url{https://nginx.org/en/}, \refdate{2024年1月18日}.
    \bibitem{bib:postform-web} Mozilla Foundation, ``POST - HTTP \textbar\ MDN'', \url{https://developer.mozilla.org/ja/docs/Web/HTTP/Methods/POST}, \refdate{2024年1月22日}.
    % \bibitem{bib:} , "", \url{https://developer.mozilla.org/ja/docs/Web/HTTP/Methods/POST}, \refdate{2024年1月22日}.
    
    
    % \bibitem{bib:Ishizaka}
    % 石坂守,山口賢一,岩田大志:``スキャン非同期記憶素子およびそれを備えた半導体集積回路ならびにその設計方法およびテストパターン'', 電子情報通信学会論文A,Vol.j102-A,No.6,pp.172-181,Jun.(2019)

    % \bibitem{textbook} \UTF{9AD9}玉 圭樹, "マルエージェント学習 -- 相互作用の謎に迫る --", 株式会社コロナ社, 2003年4月23日.
    % \bibitem{book} author, "title", publish, 2020.
    % \bibitem{webpage} author, "title", \url{URL}, \refdate{2021年8月15日}.
\end{thebibliography}

% なるべく書籍がいい
% 念の為の公式のテキストもメモ
% 
% * docker
% 本がある
% * docker-compose
% 本がある
% * Node.js
% * Leaflet.js
% https://leafletjs.com
% * PushAPI
% https://www.w3.org/TR/push-api/
% https://www.w3.org/standards/types/#WD
% https://developer.mozilla.org/en-US/docs/Web/API/Push_API
% * Nginx
% * Reverse Proxy

\expandafter\ifx\csname ifdraft\endcsname\relax
    % ----- 仕掛け 開始 -----
    \newcommand{\ifdraft}{false}
    % ----- 仕掛け 終了 -----
    % \include{references.tex}
    \end{document}
\fi
    \end{document}
\fi

%! Tex root = ../main.tex
\expandafter\ifx\csname ifdraft\endcsname\relax
    \documentclass[12pt]{honka_v1}
    \usepackage{soturon}
    \begin{document}
\fi
% 以下本文
\section{システム構成}
\subsection{全体構成}
本研究で作成するシステム全体のシステム図を\reff{fig:all-system}に示す.
まず,畑内に設置した赤外線センサノード群で動物の侵入を検知し,
そのデータをワイヤレス通信のLoRa通信を用いてゲートウェイに集約する.
ゲートウェイは集約されたデータをLTE通信を用いてInternet経由で
クラウドサーバにデータを送信する.
クラウドサーバではデータの管理やデータの解析,
農業従事者への通知および解析結果の表示を行う.
本論文ではクラウドサーバの解析アルゴリズム以外を担うため,
次節でクラウドサーバの構成と詳細について述べる.

\myfigure{./images/all-system.pdf}{全体のシステム図}{fig:all-system}

\subsection{サーバ構成}

本システムのクラウドサーバは\reff{fig:all-system}にある通り,
以下の4つの役割の担い,それぞれの役割が連携しながら処理を行う.
\begin{itemize}
    \item データベース
    \item 解析ノード
    \item 通知サーバ
    \item Webアプリケーション
\end{itemize}

クラウドサーバの役割が4つあり,それぞれが連携するため処理の流れが複雑になっている.
クラウドサーバの処理を整理するためにDFD(Data Flow Diagram)を\reff{fig:srv-dfd0}と\reff{fig:srv-dfd}に示す.
クラウドサーバの処理には
畑のデータを送ってくるゲートウェイ,
本システムに登録して通知や解析結果を受け取る農業従事者,
センサデータや解析データ,農業従事者の登録者情報を保存するデータベース
が関わっている.
ゲートウェイからは畑で検知したセンサデータがクラウドサーバへ一方的に送信される.
クラウドサーバでは送信されてきたセンサデータをデータベースに一時的に保存する.
保存したセンサデータを用いて解析を行い,通知と解析後データの保存を行う.
通知では事前に登録した農業従事者の登録者情報からPush通知を行う.
また,解析後データから行動履歴表示を行うWebアプリケーションを農業従事者へ提供する.

% ゲートウェイからは農場で得られた赤外線センサのデータがクラウドサーバへ一方的に送信されてくる.
% 農業従事者については畑の情報や認証用の本人情報などを含んだ登録者情報を受け取り,
% 畑に動物の侵入があったことを知らせるPush通知と
% センサデータを解析した結果得られた行動軌跡についての情報をまとめた行動履歴を提供する.
% データベースには畑から得られたセンサデータおよび解析後データと農業従事者から受け取った登録者情報を保管する.

\myfigures{./images/server-dfd0.pdf}{クラウドサーバのDFD(レベル0)}{fig:srv-dfd0}{6}
\myfigures{./images/server-dfd.pdf}{クラウドサーバのDFD(レベル1)}{fig:srv-dfd}{9}

クラウドサーバのDFD(\reff{fig:srv-dfd0}と\reff{fig:srv-dfd})を参考にクラウドサーバ上の構成を決める.
本システムのクラウドサーバは文字通り,クラウド上にシステムを設置するため,
開発環境(手元のPC)とのOSの違いといった環境差が生じる.
そういった環境差を最小限にするためにDockerを用いてDockerコンテナ内に各システムを構築する.
また,クラウド内の4つのシステムを連携させるためにDocker Composeを用いる.

Docker Composeを用いて作成するクラウドサーバの内部ネットワークを\reff{fig:srv-network}に示す.
クラウドサーバ内部に作成するDockerコンテナは以下の6つである.
また,Dockerコンテナごとの詳細については後述する.
\begin{itemize}
    \item リバースプロキシ(Reverse Proxy)
    \item 通知サーバ(Notification-app)
    \item Webアプリケーション(Visualizer-app)
    \item データベース\begin{itemize}
        \item データベースのアクセス制御(Database-API)
        \item データベース本体(Database)
    \end{itemize}
    \item 解析ノード(Analyzer-node)
\end{itemize}

また,Dockerコンテナ間などの通信を行うためにDockerネットワークを以下の3つ用意する.
\begin{itemize}
    \item surface
    \item inside
    \item db-bus
\end{itemize}
surfaceネットワークではInternetと接続するDockerコンテナのみ所属させる.
insideネットワークには基本的なDockerコンテナ全て所属させる.
db-busネットワークにはデータベース管理の処理を担うDockerコンテナのみ所属させる.

\myfigure{./images/server-network.pdf}{クラウドサーバのネットワーク図}{fig:srv-network}



% ============================
\subsubsection{リバースプロキシ(Reverse Proxy)}
クラウドサーバには前述の4つの役割があり,それぞれについてDockerコンテナが作成される.
しかし,クラウドサーバ内に4つのサーバを作成するとなると,
ポートの衝突などを回避するためにDockerコンテナごとに設定を細かく変える必要が出てくる.
細かい設定が増えるとクラウドサーバ全体の設定を把握しづらくなり管理保守することが難しくなる.
そこでリバースプロキシを用いてInternetとクラウドサーバ内部のDockerコンテナを仲介・制御する.
リバースプロキシは外からのアクセスの種類によって内部のどのサーバと通信するかを制御する機能を持ったサーバのことである.
リバースプロキシによって,クラウドサーバ内部に存在する複数機能を担う各Dockerコンテナへのアクセスを
まとめて管理する.

実装するリバースプロキシの具体的な仕様としては,
クラウドサーバにくる全てのデータアクセス(HTTP通信)を役割ごとに適当なDockerコンテナに通信するため,
surfaceネットワークとinsideネットワークに所属する.
% リバースプロキシではサーバにくる全てのデータアクセス(HTTP通信)を役割ごとに適当なDockerコンテナに通信するように制御を行い,
% クラウドサーバの門番的な役割を担う.
% そのため,Dockerネットワークとしてはsurfaceネットワークとinsideネットワークに所属する.

NginxのDockerイメージをベースとし,/etc/nginx/nginx.confを編集することで
リバースプロキシサーバとしてDockerコンテナを作成する.

リバースプロキシで行う転送処理は以下の通りである.

\begin{description}
    \item[/notification :]    通知サーバ(Notification-app)
    \item[/maps :]            Webアプリケーション(Visualizer-app)
    \item[/nodegw (POST) :]   データベース(Database-API)
\end{description}



% ============================
\subsubsection{通知サーバ(Notification-app)}
通知サーバではレベル1のDFD\reff{fig:srv-dfd}にある「4. 通知」で行うため.Push通知を用いて農業従事者への通知を実現する.
まず,Push通知はスマートフォンやパソコンなどに搭載されている通知方法の一つで,
画面を付けていない状態や別のアプリを利用中といった場面でも通知音やバナーといった方法で即時通知することができる.
このような常に通知用の管理画面を開く必要がないことや即時性に着目して,
通知をPush通知を用いて実現することにした.
しかし基本的なPush通知の実装はその端末にあったアプリケーションを作成する必要があり,
AndroidやiOS,Windows,MacOSといった対応させたいOSの数だけアプリケーションを作成する必要がある.
内部構造が基本的に同じになるとは言ってもOSごとのアプリケーションを作成するのはコストが高く,
バージョンアップといった保守の手間も非常に大きい.

そこで,WebPushというものを活用する.
WebPushはブラウザを用いたPush通知を行う仕組みであり,一つのWebサーバを作成するだけで
対応するブラウザを介して複数種類の端末でPush通知を実現することができる.
またWebPushは基本的なブラウザに対応しているため,簡単に複数種類の端末へのPush通知が実現することが可能である.


したがって,通知サーバでは登録された畑に動物の侵入があったことをWebPushを用いて農業従事者に通知するサーバを実現する.
具体的には
WebPushでVAPIDを用いてサーバの登録を行い,
Node.jsのDockerイメージをベースとしてWebPushを行うサーバを実現する.

通知サーバはレベル1のDFD(\reff{fig:srv-dfd})においては「4. 通知」を主に担う.
そのため「2. 解析」からの通知シグナルとデータベースから登録者情報を取得するために
DockerネットワークでAnalyzer-nodeおよびDatabase-APIと同じinsideに所属する.



% ============================
\subsubsection{Webアプリケーション(Visualizer-app)}
Webアプリケーションではレベル1のDFD\reff{fig:srv-dfd}にある「3. 登録」と「5. 行動履歴表示」を行う.
行動履歴表示では特に侵入した動物の侵入経路の分かりやすさが重要である.
そこでLeaflet.jsというマップライブラリを用いる.
Lealfet.jsは地図の拡大縮小やマーカの表示など基本的な地図アプリの機能に加えて,
線や多角形領域,ピンといったマーカを地図に投影することができるため採用する.
加えて,WebページとしてJavaScript(Leaflet.js)を利用するため,
JavaScriptのNode.jsを用いることで利用する言語を統一し,管理を容易にする.

また,
本システムは複数の農業従事者および畑を対象としてサービスを提供するため,
どこの畑で,誰が管理者であるかを登録する必要がある.
登録ページについて開発中のページを\reff{fig:map-register}に示す.

\myfigure{./images/map-register.png}{開発中の登録ページ}{fig:map-register}

% Webアプリケーションでは畑の登録と解析データから得られた行動履歴の表示を実現する.
% Webアプリケーションでは畑など地図の表示および図形の投影が必要なため,
Webアプリケーションの具体的な実装としては,
Node.jsのDockerイメージをベースとしてHTMLやCSS,Leaflet.jsを含めたJavaScriptを用いて実現する.
Webアプリケーションはレベル1のDFD(\reff{fig:srv-dfd})においては「3. 登録」と「5. 行動履歴表示」
を主に担う.
そのためデータベースへの登録者情報の保存および解析後データの取得するため,
DockerネットワークでDatabase-APIと同じinsideに所属する.



% ============================
\subsubsection{データベース(Database, Database-API)}
データベースではMySQLを用いたデータベースコンテナと
Node.jsを用いたデータベースコンテナ管理用のAPIコンテナの2つで構成する.
APIコンテナでデータベースアクセスを仲介することで,
データベースアクセスの簡易化や誤操作の抑制を実現する.

Dockerネットワークとしてはdb-busネットワークに所属して,
他のDockerコンテナからデータベースアクセスを受けるためにDatabase-APIコンテナのみ
insideネットワークにも所属する.


% ============================
\subsubsection{データ解析ノード(Analyzer-node)}
データ解析ノードではpythonイメージを主体としてデータベースに蓄えられる赤外線センサのデータを解析する役割を持つ.
本コンテナではDatabase-APIコンテナと通信してデータの取得および解析後のデータの保存を行い,通知サーバに通知シグナルを送信する.
そのため,Dockerネットワークとしてはinsideネットワークに所属する.

解析の詳細に関しては本論文範囲外であるため割愛する.





% ソースコードの記述
% \mylisting{ファイルパス}{caption}{label}
% \mylisting{report.sty}{レポートのスタイルファイル}{fig:style}

% 図の配置
% \myfigure{ファイルパス}{caption}{label} サイズが画面横の10割
% \myfigureh{ファイルパス}{caption}{label} サイズが画面縦の10割
% \myfigurem{ファイルパス}{caption}{label} サイズが画面横の6割
% \myfigures{ファイルパス}{caption}{label}{size} サイズが画面横の{size}割

% \myfigure{figure.png}{画像ファイル}{fig:image}
% \reff{fig:image}

% 式の作成
% \begin{align}
%     \label{for:opamp_inout_char}
%     V_o = A \times (V_+ - V_-)
% \end{align}
% \refe{for:opamp_inout_char}

% 表の作成
% \reft{tb:}
% \begin{table}[htbp]
%     \caption{}
%     \label{tb:}
%     \centering
%     \begin{tabular}{|c|c|c|c|c|c|}\hline
%         器具名 & 個数 & メーカー & 型番 & シリアル番号 & 管理番号 \\ \hline
%         \hline
%         TTL-IC用直流電源 & 1台 & --- & --- & --- & 2班 \\ \hline
%     \end{tabular}
% \end{table}

% 箇条書き
% \begin{itemize}
%     \item 
% \end{itemize}

\expandafter\ifx\csname ifdraft\endcsname\relax
    % ----- 仕掛け 開始 -----
    \newcommand{\ifdraft}{false}
    % ----- 仕掛け 終了 -----
    %! Tex root = ../main.tex
\expandafter\ifx\csname ifdraft\endcsname\relax
    \documentclass[12pt]{honka_v1}
    \usepackage{soturon}
    \begin{document}
\fi
% 以下本文
% \printbibliography[title=参考文献]
% 参考文献をchapterとして目次に追加
% \addcontentsline{toc}{chapter}{参考文献}
\begin{thebibliography}{99}
    \bibitem{bib:jugaihigai-web} 農林水産省, ``野生鳥獣による農作物被害状況の推移'', \url{https://www.maff.go.jp/j/seisan/tyozyu/higai/hogai_zyoukyou/attach/pdf/index-31.pdf}, \refdate{2024年1月18日}.
    
    \bibitem{bib:docker-book} 大澤文孝, 浅井尚, ``触って学ぶクラウドインフラ docker 基礎からのコンテナ構築'', 日経BPマーケティング, 2020年.
    
    \bibitem{bib:node-book} 掌田津耶乃, ``Node.js 超入門'', 株式会社 秀和システム, 2017年.
    \bibitem{bib:express-web} StrongLoop, IBM, ``Express - Node.js web application framework'', \url{https://expressjs.com/}, \refdate{2024年1月18日}.
    \bibitem{bib:leafletjs-web} Volodymyr Agafonkin, ``Leaflet - a JavaScript library for interactive maps'', \url{https://leafletjs.com}, \refdate{2024年1月18日}.
    
    \bibitem{bib:pushapi-web} Mozilla Foundation, ``プッシュ API - Web API \textbar\ MDN'', \url{https://developer.mozilla.org/ja/docs/Web/API/Push_API}, \refdate{2024年1月18日}.
    \bibitem{bib:webpush-rfc} M. Thomson, E. Damaggio, B. Raymor, Ed., ``Generic Event Delivery Using HTTP Push'', \url{https://datatracker.ietf.org/doc/html/rfc8030}, RFC8030, December 2016, \refdate{2024年1月21日}.
    \bibitem{bib:serviceworker-web} Mozilla Foundation, ``サービスワーカー API - Web API \textbar\ MDN'', \url{https://developer.mozilla.org/ja/docs/Web/API/Service_Worker_API}, \refdate{2024年1月18日}.
    \bibitem{bib:vapid-rfc} M. Thomson, P. Beverloo, ``Voluntary Application Server Identification (VAPID)'', \url{https://datatracker.ietf.org/doc/html/rfc8292}, RFC8292, November 2017, \refdate{2024年1月21日}.
    
    \bibitem{bib:nginx-web} Nginx, ``nginx'', \url{https://nginx.org/en/}, \refdate{2024年1月18日}.
    \bibitem{bib:postform-web} Mozilla Foundation, ``POST - HTTP \textbar\ MDN'', \url{https://developer.mozilla.org/ja/docs/Web/HTTP/Methods/POST}, \refdate{2024年1月22日}.
    % \bibitem{bib:} , "", \url{https://developer.mozilla.org/ja/docs/Web/HTTP/Methods/POST}, \refdate{2024年1月22日}.
    
    
    % \bibitem{bib:Ishizaka}
    % 石坂守,山口賢一,岩田大志:``スキャン非同期記憶素子およびそれを備えた半導体集積回路ならびにその設計方法およびテストパターン'', 電子情報通信学会論文A,Vol.j102-A,No.6,pp.172-181,Jun.(2019)

    % \bibitem{textbook} \UTF{9AD9}玉 圭樹, "マルエージェント学習 -- 相互作用の謎に迫る --", 株式会社コロナ社, 2003年4月23日.
    % \bibitem{book} author, "title", publish, 2020.
    % \bibitem{webpage} author, "title", \url{URL}, \refdate{2021年8月15日}.
\end{thebibliography}

% なるべく書籍がいい
% 念の為の公式のテキストもメモ
% 
% * docker
% 本がある
% * docker-compose
% 本がある
% * Node.js
% * Leaflet.js
% https://leafletjs.com
% * PushAPI
% https://www.w3.org/TR/push-api/
% https://www.w3.org/standards/types/#WD
% https://developer.mozilla.org/en-US/docs/Web/API/Push_API
% * Nginx
% * Reverse Proxy

\expandafter\ifx\csname ifdraft\endcsname\relax
    % ----- 仕掛け 開始 -----
    \newcommand{\ifdraft}{false}
    % ----- 仕掛け 終了 -----
    % \include{references.tex}
    \end{document}
\fi
    \end{document}
\fi

%! Tex root = ../main.tex
\expandafter\ifx\csname ifdraft\endcsname\relax
    \documentclass[12pt]{honka_v1}
    \usepackage{soturon}
    \begin{document}
\fi
% 以下本文
\section{検証実験}
\subsection{実験方法}
\subsubsection{ゲートウェイのLTEモジュールとの疎通実験}
ゲートウェイが使用するLTE通信モジュールとクラウドサーバ間での
POST形式での通信が可能か確認する.

実験手順としてはゲートウェイに搭載予定のLTEモジュールからInternetを経由させて
用意したサーバにデータを送信し,通信を確認する.
LTEモジュールから送信は
POST形式でHTTP通信を用いて行う.
また,送るデータはJSON形式で送信し,データ内容としては\reft{tb:lte-json}に示す.

\begin{table}[htbp]
    \caption{LTE通信するJSONデータ}
    \label{tb:lte-json}
    \centering
    \begin{tabular}{c|c|l}\hline
        データ名 & データフォーマット & 注釈 \\
        \hline
        \multirow{2}{*}{timestamp} & 文字列 & 検出時刻 \\
        & {\footnotesize(YYYYMMDDHHMMSSSSS)} & {\footnotesize(年月日,時分秒[ms])} \\
        node & 数値 & センサノード番号 \\
        dir & 数値 & ノード内のセンサ番号 \\
        sd & (任意長) & センサデータ{\footnotesize(サンプリングデータ)} \\
        \hline
    \end{tabular}
\end{table}


\subsubsection{通知サーバのWebPush実験}
クラウドサーバの構成要素である通知サーバ(Notification-app)から
WebPushによる通知ができるかどうかを確認する.

実験手順としては以下の通りである.
\begin{enumerate}
    \item Docker Composeでサーバを起動する.
    \item ブラウザアプリで起動したサーバにアクセスする.
    \item アクセス時に通知を受け取るかの確認が出るため,それを許可する.
    \item サーバ側からWebPushでPush通知を送る.
    \item ブラウザアプリを通じて通知がくるかを確認する.
\end{enumerate}

\subsubsection{Webアプリケーションの行動履歴表示実験}
クラウドサーバの構成要素であるWebアプリケーション(Visualizer-app)の
行動履歴表示ができるかを確認する.

実験内容としては,
ダミーで用意した畑の位置情報,センサノード位置情報,経路情報について
正しく表示できているかをWebページにアクセスして確認する.
ダミーで用意したデータについては\reft{tb:damy-history}に示す.
\begin{table}[htbp]
    \caption{ダミーデータ}
    \label{tb:damy-history}
    \centering
    \begin{tabular}{c|cc|c}\hline
        データ種類 & 緯度 & 経度 & 注釈\\
        \hline
        畑位置 & 34.64801074823137 & 135.75705349445346 & 奈良高専の校庭\\
        \hline
        センサ1 & 34.64746793595177 & 135.75796544551852 & 畑の南東\\
        センサ2 & 34.64801074823137 & 135.75796544551852 & 畑の東\\
        センサ3 & 34.64852707856505 & 135.75796544551852 & 畑の北東\\
        センサ4 & 34.64852707856505 & 135.75705349445346 & 畑の北\\
        センサ5 & 34.64852707856505 & 135.75623810291293 & 畑の北西\\
        センサ6 & 34.64801074823137 & 135.75623810291293 & 畑の西\\
        センサ7 & 34.64746793595177 & 135.75623810291293 & 畑の南西\\
        センサ8 & 34.64746793595177 & 135.75705349445346 & 畑の南\\
        \hline
        \multirow{5}{*}{経路情報}
        & 34.64879186210419 & 135.75744509696963 & 1番目の位置 \\
        & 34.64826229418025 & 135.75744509696963 & 2番目の位置 \\
        & 34.64828435957796 & 135.75665116310122 & 3番目の位置 \\
        & 34.64780333257661 & 135.75666189193728 & 4番目の位置 \\
        & 34.64724286640332 & 135.75664043426517 & 5番目の位置 \\
        \hline
    \end{tabular}
\end{table}

\subsection{実験結果}
\subsubsection{ゲートウェイのLTEモジュールとの疎通実験}
ゲートウェイとのLTE通信を受信した結果を\reff{fig:lte-post-log}と\reff{fig:lte-post-body}に示す.

\myfigure{./images/lte-post-log.png}{サーバの受信ログ}{fig:lte-post-log}
\myfigure{./images/lte-post.png}{サーバの受信データ(body情報の一部抜粋)}{fig:lte-post-body}

\reff{fig:lte-post-log}にあるように,「\{``timestamp'':``20240123135017200'',``node'':``1'',``dir'':``1'',\\
``sd'':``1234567890''\}」
というデータを受け取ったことがわかる.
% (それ以外の「\{``node'':``1'',``dir'':``1'',``sd'':``1234567890''\}」と
% 「\{``timeStamp'':``Jikan'',``NodeID'':\\``ID'',``Direction'':``dir'',``sampledata'':``data dayo!''\}」,
% 「\{``name'':``andesite'',``age'':\\``20''\}」
% はサーバが正しく起動しているかどうかの
% チェックとしてcurlコマンドやJavascriptを用いて送ったものである.)
また,\reff{fig:lte-post-body}より
「methods: \{ post: true \}」のようにPOST形式で送られていることがわかる.
加えて,「[Symbol(kHesaders)]」の項目から
「content-length」は54,
「content-type」は`application/x-www-form-urlencoded'
であることがわかる.
(「host」の伏せてある部分はクラウドサーバのホスト名である.)

「content-length」はbodyの波括弧(\{\})と空白スペース,クォーテーションを除いた
数である54文字と一致するため,bodyのデータサイズと考えられる.
また,「content-type」の`application/x-www-form-urlencoded'は
HTMLフォームでデータを送信した時と同じタイプであるため\cite{bib:postform-web},
ゲートウェイからの通信はHTMLフォームによるアクセスと同義に扱えると考えられる.

\subsubsection{通知サーバのWebPush実験}
通知サーバの通知結果を\reff{fig:alart-check}と\reff{fig:alart-webpush}
に示す.今回の実験ではPC版のGoogle Chrome(バージョン:120.0.6099.234(Official Build) (x86\_64))
をブラウザとして利用した.

\myfigurem{./images/alart-check.png}{WebPushの通知許可への確認ダイアログ}{fig:alart-check}
\myfigurem{./images/alart-push.png}{WebPushの通知}{fig:alart-webpush}

\reff{fig:alart-check}はWebPushを実施するアプリケーションサーバにアクセスした時の
ダイアログである.
\reff{fig:alart-check}の選択肢で「許可する」を選択することで,WebPushの通知が受け取れるようになる.
「許可する」を選択後,サーバにWebPush送信用のメソッドを起動するようにシグナルを送ると,
\reff{fig:alart-webpush}がデスクトップに表示された.
\reff{fig:alart-webpush}の通り,Google Chromeからの通知であることがわかり,
Push通知をクリックするとGoogle Chromeに遷移させられたため,Google ChromeによるPush通知だと判断できる.


\subsubsection{Webアプリケーションの行動履歴表示実験}
Webアプリケーションの行動履歴表示について\reft{tb:damy-history}を表示した
Webページを\reff{fig:map-history}に示す.

\myfigure{./images/map-history.png}{Webアプリケーションの行動履歴表示(ダミーデータ)}{fig:map-history}

\reff{fig:map-history}より,
奈良高専の校庭を中心とした
8つのセンサノード(ピン)と,
行動経路(青い折れ線)が描画されていることがわかる.
そのため正しくダミーデータ(\reft{tb:damy-history})通りの描画が行えていると判断できる.


\subsection{まとめ}
本章の検証実験により,
ゲートウェイとの疎通確認および通信方法の特定,
通知サーバによるWebPushの通知の確認,
Webアプリケーションによる行動履歴表示の確認
が行うことができた.

本論文ではできなかったがシステムとして効果を確認するためには,
赤外線センサノードおよびゲートウェイを実際の畑に設置し,
動物が畑に侵入した時の実際のセンサデータを取得,
クラウドサーバへ送信したのち,
行動解析アルゴリズムを用いて侵入した動物の行動を推定,
そうして得られたデータからPCやスマートフォンに対してWebPushによる通知,
ブラウザを通した行動履歴の表示する,
という一連の流れを検証する必要がある.
% また,本論文の担当外の機能との連携に伴い,
% クラウドサーバの機能としては,
% データ解析ノードの「データ取得」,「データ解析」,「通知シグナル」という
% 一連の流れを実現する方法を考慮する必要がある.


% ソースコードの記述
% \mylisting{ファイルパス}{caption}{label}
% \mylisting{report.sty}{レポートのスタイルファイル}{fig:style}

% 図の配置
% \myfigure{ファイルパス}{caption}{label} サイズが画面横の10割
% \myfigureh{ファイルパス}{caption}{label} サイズが画面縦の10割
% \myfigurem{ファイルパス}{caption}{label} サイズが画面横の6割
% \myfigures{ファイルパス}{caption}{label}{size} サイズが画面横の{size}割

% \myfigure{figure.png}{画像ファイル}{fig:image}
% \reff{fig:image}

% 式の作成
% \begin{align}
%     \label{for:opamp_inout_char}
%     V_o = A \times (V_+ - V_-)
% \end{align}
% \refe{for:opamp_inout_char}

% 表の作成
% \reft{tb:}
% \begin{table}[htbp]
%     \caption{}
%     \label{tb:}
%     \centering
%     \begin{tabular}{|c|c|c|c|c|c|}\hline
%         器具名 & 個数 & メーカー & 型番 & シリアル番号 & 管理番号 \\ \hline
%         \hline
%         TTL-IC用直流電源 & 1台 & --- & --- & --- & 2班 \\ \hline
%     \end{tabular}
% \end{table}

% 箇条書き
% \begin{itemize}
%     \item 
% \end{itemize}

\expandafter\ifx\csname ifdraft\endcsname\relax
    % ----- 仕掛け 開始 -----
    \newcommand{\ifdraft}{false}
    % ----- 仕掛け 終了 -----
    %! Tex root = ../main.tex
\expandafter\ifx\csname ifdraft\endcsname\relax
    \documentclass[12pt]{honka_v1}
    \usepackage{soturon}
    \begin{document}
\fi
% 以下本文
% \printbibliography[title=参考文献]
% 参考文献をchapterとして目次に追加
% \addcontentsline{toc}{chapter}{参考文献}
\begin{thebibliography}{99}
    \bibitem{bib:jugaihigai-web} 農林水産省, ``野生鳥獣による農作物被害状況の推移'', \url{https://www.maff.go.jp/j/seisan/tyozyu/higai/hogai_zyoukyou/attach/pdf/index-31.pdf}, \refdate{2024年1月18日}.
    
    \bibitem{bib:docker-book} 大澤文孝, 浅井尚, ``触って学ぶクラウドインフラ docker 基礎からのコンテナ構築'', 日経BPマーケティング, 2020年.
    
    \bibitem{bib:node-book} 掌田津耶乃, ``Node.js 超入門'', 株式会社 秀和システム, 2017年.
    \bibitem{bib:express-web} StrongLoop, IBM, ``Express - Node.js web application framework'', \url{https://expressjs.com/}, \refdate{2024年1月18日}.
    \bibitem{bib:leafletjs-web} Volodymyr Agafonkin, ``Leaflet - a JavaScript library for interactive maps'', \url{https://leafletjs.com}, \refdate{2024年1月18日}.
    
    \bibitem{bib:pushapi-web} Mozilla Foundation, ``プッシュ API - Web API \textbar\ MDN'', \url{https://developer.mozilla.org/ja/docs/Web/API/Push_API}, \refdate{2024年1月18日}.
    \bibitem{bib:webpush-rfc} M. Thomson, E. Damaggio, B. Raymor, Ed., ``Generic Event Delivery Using HTTP Push'', \url{https://datatracker.ietf.org/doc/html/rfc8030}, RFC8030, December 2016, \refdate{2024年1月21日}.
    \bibitem{bib:serviceworker-web} Mozilla Foundation, ``サービスワーカー API - Web API \textbar\ MDN'', \url{https://developer.mozilla.org/ja/docs/Web/API/Service_Worker_API}, \refdate{2024年1月18日}.
    \bibitem{bib:vapid-rfc} M. Thomson, P. Beverloo, ``Voluntary Application Server Identification (VAPID)'', \url{https://datatracker.ietf.org/doc/html/rfc8292}, RFC8292, November 2017, \refdate{2024年1月21日}.
    
    \bibitem{bib:nginx-web} Nginx, ``nginx'', \url{https://nginx.org/en/}, \refdate{2024年1月18日}.
    \bibitem{bib:postform-web} Mozilla Foundation, ``POST - HTTP \textbar\ MDN'', \url{https://developer.mozilla.org/ja/docs/Web/HTTP/Methods/POST}, \refdate{2024年1月22日}.
    % \bibitem{bib:} , "", \url{https://developer.mozilla.org/ja/docs/Web/HTTP/Methods/POST}, \refdate{2024年1月22日}.
    
    
    % \bibitem{bib:Ishizaka}
    % 石坂守,山口賢一,岩田大志:``スキャン非同期記憶素子およびそれを備えた半導体集積回路ならびにその設計方法およびテストパターン'', 電子情報通信学会論文A,Vol.j102-A,No.6,pp.172-181,Jun.(2019)

    % \bibitem{textbook} \UTF{9AD9}玉 圭樹, "マルエージェント学習 -- 相互作用の謎に迫る --", 株式会社コロナ社, 2003年4月23日.
    % \bibitem{book} author, "title", publish, 2020.
    % \bibitem{webpage} author, "title", \url{URL}, \refdate{2021年8月15日}.
\end{thebibliography}

% なるべく書籍がいい
% 念の為の公式のテキストもメモ
% 
% * docker
% 本がある
% * docker-compose
% 本がある
% * Node.js
% * Leaflet.js
% https://leafletjs.com
% * PushAPI
% https://www.w3.org/TR/push-api/
% https://www.w3.org/standards/types/#WD
% https://developer.mozilla.org/en-US/docs/Web/API/Push_API
% * Nginx
% * Reverse Proxy

\expandafter\ifx\csname ifdraft\endcsname\relax
    % ----- 仕掛け 開始 -----
    \newcommand{\ifdraft}{false}
    % ----- 仕掛け 終了 -----
    % \include{references.tex}
    \end{document}
\fi
    \end{document}
\fi

% %! Tex root = ../main.tex
\expandafter\ifx\csname ifdraft\endcsname\relax
    \documentclass[12pt]{honka_v1}
    \usepackage{soturon}
    \begin{document}
\fi
% 以下本文
\section{考察}
% 考察・検討を書く

% ソースコードの記述
% \mylisting{ファイルパス}{caption}{label}
% \mylisting{report.sty}{レポートのスタイルファイル}{fig:style}

% 図の配置
% \myfigure{ファイルパス}{caption}{label} サイズが画面横の10割
% \myfigureh{ファイルパス}{caption}{label} サイズが画面縦の10割
% \myfigurem{ファイルパス}{caption}{label} サイズが画面横の6割
% \myfigures{ファイルパス}{caption}{label}{size} サイズが画面横の{size}割

% \myfigure{figure.png}{画像ファイル}{fig:image}
% \reff{fig:image}

% 式の作成
% \begin{align}
%     \label{for:opamp_inout_char}
%     V_o = A \times (V_+ - V_-)
% \end{align}
% \refe{for:opamp_inout_char}

% 表の作成
% \reft{tb:}
% \begin{table}[htbp]
%     \caption{}
%     \label{tb:}
%     \centering
%     \begin{tabular}{|c|c|c|c|c|c|}\hline
%         器具名 & 個数 & メーカー & 型番 & シリアル番号 & 管理番号 \\ \hline
%         \hline
%         TTL-IC用直流電源 & 1台 & --- & --- & --- & 2班 \\ \hline
%     \end{tabular}
% \end{table}

% 箇条書き
% \begin{itemize}
%     \item 
% \end{itemize}

\expandafter\ifx\csname ifdraft\endcsname\relax
    % ----- 仕掛け 開始 -----
    \newcommand{\ifdraft}{false}
    % ----- 仕掛け 終了 -----
    %! Tex root = ../main.tex
\expandafter\ifx\csname ifdraft\endcsname\relax
    \documentclass[12pt]{honka_v1}
    \usepackage{soturon}
    \begin{document}
\fi
% 以下本文
% \printbibliography[title=参考文献]
% 参考文献をchapterとして目次に追加
% \addcontentsline{toc}{chapter}{参考文献}
\begin{thebibliography}{99}
    \bibitem{bib:jugaihigai-web} 農林水産省, ``野生鳥獣による農作物被害状況の推移'', \url{https://www.maff.go.jp/j/seisan/tyozyu/higai/hogai_zyoukyou/attach/pdf/index-31.pdf}, \refdate{2024年1月18日}.
    
    \bibitem{bib:docker-book} 大澤文孝, 浅井尚, ``触って学ぶクラウドインフラ docker 基礎からのコンテナ構築'', 日経BPマーケティング, 2020年.
    
    \bibitem{bib:node-book} 掌田津耶乃, ``Node.js 超入門'', 株式会社 秀和システム, 2017年.
    \bibitem{bib:express-web} StrongLoop, IBM, ``Express - Node.js web application framework'', \url{https://expressjs.com/}, \refdate{2024年1月18日}.
    \bibitem{bib:leafletjs-web} Volodymyr Agafonkin, ``Leaflet - a JavaScript library for interactive maps'', \url{https://leafletjs.com}, \refdate{2024年1月18日}.
    
    \bibitem{bib:pushapi-web} Mozilla Foundation, ``プッシュ API - Web API \textbar\ MDN'', \url{https://developer.mozilla.org/ja/docs/Web/API/Push_API}, \refdate{2024年1月18日}.
    \bibitem{bib:webpush-rfc} M. Thomson, E. Damaggio, B. Raymor, Ed., ``Generic Event Delivery Using HTTP Push'', \url{https://datatracker.ietf.org/doc/html/rfc8030}, RFC8030, December 2016, \refdate{2024年1月21日}.
    \bibitem{bib:serviceworker-web} Mozilla Foundation, ``サービスワーカー API - Web API \textbar\ MDN'', \url{https://developer.mozilla.org/ja/docs/Web/API/Service_Worker_API}, \refdate{2024年1月18日}.
    \bibitem{bib:vapid-rfc} M. Thomson, P. Beverloo, ``Voluntary Application Server Identification (VAPID)'', \url{https://datatracker.ietf.org/doc/html/rfc8292}, RFC8292, November 2017, \refdate{2024年1月21日}.
    
    \bibitem{bib:nginx-web} Nginx, ``nginx'', \url{https://nginx.org/en/}, \refdate{2024年1月18日}.
    \bibitem{bib:postform-web} Mozilla Foundation, ``POST - HTTP \textbar\ MDN'', \url{https://developer.mozilla.org/ja/docs/Web/HTTP/Methods/POST}, \refdate{2024年1月22日}.
    % \bibitem{bib:} , "", \url{https://developer.mozilla.org/ja/docs/Web/HTTP/Methods/POST}, \refdate{2024年1月22日}.
    
    
    % \bibitem{bib:Ishizaka}
    % 石坂守,山口賢一,岩田大志:``スキャン非同期記憶素子およびそれを備えた半導体集積回路ならびにその設計方法およびテストパターン'', 電子情報通信学会論文A,Vol.j102-A,No.6,pp.172-181,Jun.(2019)

    % \bibitem{textbook} \UTF{9AD9}玉 圭樹, "マルエージェント学習 -- 相互作用の謎に迫る --", 株式会社コロナ社, 2003年4月23日.
    % \bibitem{book} author, "title", publish, 2020.
    % \bibitem{webpage} author, "title", \url{URL}, \refdate{2021年8月15日}.
\end{thebibliography}

% なるべく書籍がいい
% 念の為の公式のテキストもメモ
% 
% * docker
% 本がある
% * docker-compose
% 本がある
% * Node.js
% * Leaflet.js
% https://leafletjs.com
% * PushAPI
% https://www.w3.org/TR/push-api/
% https://www.w3.org/standards/types/#WD
% https://developer.mozilla.org/en-US/docs/Web/API/Push_API
% * Nginx
% * Reverse Proxy

\expandafter\ifx\csname ifdraft\endcsname\relax
    % ----- 仕掛け 開始 -----
    \newcommand{\ifdraft}{false}
    % ----- 仕掛け 終了 -----
    % \include{references.tex}
    \end{document}
\fi
    \end{document}
\fi

%! Tex root = ../main.tex
\expandafter\ifx\csname ifdraft\endcsname\relax
    \documentclass[12pt]{honka_v1}
    \usepackage{soturon}
    \begin{document}
\fi
% 以下本文
\section{あとがき}
% \section{展望}
% 卒業研究の成果を総括して述べ,研究目標に対してどこまで到達できたかを吟味,検討し,
% その研究で成し得なかったことや研究の将来性に関することを記述する.

本論文では害獣検出と行動解析を行うシステムの中で
クラウドサーバに関する部分の設計・実装・動作確認について述べてきた.
畑に設置予定のゲートウェイとのデータ通信や
通知アプリの動作検証,
Webアプリケーションによる行動履歴の表示といった,
クラウドサーバとして必要な要素についての設計・検証を行ってきたが,
まだ試験的な部分が大きく,実際に利用するには至らない部分が多い.
特に優先的に対処しなければならないこととして以下に示すことが挙げられる.
\begin{itemize}
    \item セキュリティ対策
    \item UI・UXの向上
    \item サーバの冗長性
\end{itemize}

\subsection{セキュリティ対策}
クラウドサーバではゲートウェイからの受信にHTTP通信を用いている.
そのため,データ通信が十分に暗号化されておらず,
通信中に盗み見られることや悪意を持った攻撃者がデータを装って
POST送信をしてくることが考えられる.
これに関しては
ゲートウェイに搭載できるモジュールやマイコンの性能に左右されるため,
ゲートウェイ設計担当である共同研究者とも話し合っていく必要があると考えられる.

また,クラウドサーバへのアクセスは基本的に
クラウドサーバ自体のファイアウォールやDocker Compose内のリバースプロキシ,
利用するクラウドサービス独自のセキュリティシステムを考えている.
本論文のシステムではクラウドサーバ内で
農業従事者の情報を入力し,登録するフォームを設けるなど
それらの個人情報を取り扱うこととなるため,十分に堅牢なシステムが必要だと考えられる.
そのためデータ管理や通信手段をHTTPSなどで暗号化するなど,
登録方法を別の手段を利用して本クラウドサーバ上で個人情報のやり取りを
最小限にするなどの対策することが重要である.

\subsection{UI・UXの向上}
本論文のクラウドサーバは実験結果の通り,
機能を確かめるための仮組みの設計でしかない.
そのため対象ユーザが農業従事者であり,
基本的にパソコンなどの操作に慣れていない可能性を考慮する必要がある.


\subsection{サーバの冗長性}
本クラウドサーバは常時,畑に設定したゲートウェイから
センサ情報を受け取ることが予想される.
そのため,登録される畑が増えれば
クラウドサーバへのデータ送信は肥大化していく.
特に対象としているのが夜間の獣害被害であるため,
そのデータ送信も夜間に集中する可能性がある.
そうなったときに,本論文のクラウドサーバ構成は
通信制御を行う
リバースプロキシとデータベースがボトルネックとなることが考えられる.

そのような場合に備えて将来的にはボトルネックとなる,
Dockerコンテナを複数個並列して用意することや負荷を分散させる機構(kubernetesのようなロードバランサー)
を用いることなどが必要になると考えられる.



% セキュリティやUI・UXへの考慮


% 成果の総括

% 到達度について

% 将来性





% ソースコードの記述
% \mylisting{ファイルパス}{caption}{label}
% \mylisting{report.sty}{レポートのスタイルファイル}{fig:style}

% 図の配置
% \myfigure{ファイルパス}{caption}{label} サイズが画面横の10割
% \myfigureh{ファイルパス}{caption}{label} サイズが画面縦の10割
% \myfigurem{ファイルパス}{caption}{label} サイズが画面横の6割
% \myfigures{ファイルパス}{caption}{label}{size} サイズが画面横の{size}割

% \myfigure{figure.png}{画像ファイル}{fig:image}
% \reff{fig:image}

% 式の作成
% \begin{align}
%     \label{for:opamp_inout_char}
%     V_o = A \times (V_+ - V_-)
% \end{align}
% \refe{for:opamp_inout_char}

% 表の作成
% \reft{tb:}
% \begin{table}[htbp]
%     \caption{}
%     \label{tb:}
%     \centering
%     \begin{tabular}{|c|c|c|c|c|c|}\hline
%         器具名 & 個数 & メーカー & 型番 & シリアル番号 & 管理番号 \\ \hline
%         \hline
%         TTL-IC用直流電源 & 1台 & --- & --- & --- & 2班 \\ \hline
%     \end{tabular}
% \end{table}

% 箇条書き
% \begin{itemize}
%     \item 
% \end{itemize}

\expandafter\ifx\csname ifdraft\endcsname\relax
    % ----- 仕掛け 開始 -----
    \newcommand{\ifdraft}{false}
    % ----- 仕掛け 終了 -----
    %! Tex root = ../main.tex
\expandafter\ifx\csname ifdraft\endcsname\relax
    \documentclass[12pt]{honka_v1}
    \usepackage{soturon}
    \begin{document}
\fi
% 以下本文
% \printbibliography[title=参考文献]
% 参考文献をchapterとして目次に追加
% \addcontentsline{toc}{chapter}{参考文献}
\begin{thebibliography}{99}
    \bibitem{bib:jugaihigai-web} 農林水産省, ``野生鳥獣による農作物被害状況の推移'', \url{https://www.maff.go.jp/j/seisan/tyozyu/higai/hogai_zyoukyou/attach/pdf/index-31.pdf}, \refdate{2024年1月18日}.
    
    \bibitem{bib:docker-book} 大澤文孝, 浅井尚, ``触って学ぶクラウドインフラ docker 基礎からのコンテナ構築'', 日経BPマーケティング, 2020年.
    
    \bibitem{bib:node-book} 掌田津耶乃, ``Node.js 超入門'', 株式会社 秀和システム, 2017年.
    \bibitem{bib:express-web} StrongLoop, IBM, ``Express - Node.js web application framework'', \url{https://expressjs.com/}, \refdate{2024年1月18日}.
    \bibitem{bib:leafletjs-web} Volodymyr Agafonkin, ``Leaflet - a JavaScript library for interactive maps'', \url{https://leafletjs.com}, \refdate{2024年1月18日}.
    
    \bibitem{bib:pushapi-web} Mozilla Foundation, ``プッシュ API - Web API \textbar\ MDN'', \url{https://developer.mozilla.org/ja/docs/Web/API/Push_API}, \refdate{2024年1月18日}.
    \bibitem{bib:webpush-rfc} M. Thomson, E. Damaggio, B. Raymor, Ed., ``Generic Event Delivery Using HTTP Push'', \url{https://datatracker.ietf.org/doc/html/rfc8030}, RFC8030, December 2016, \refdate{2024年1月21日}.
    \bibitem{bib:serviceworker-web} Mozilla Foundation, ``サービスワーカー API - Web API \textbar\ MDN'', \url{https://developer.mozilla.org/ja/docs/Web/API/Service_Worker_API}, \refdate{2024年1月18日}.
    \bibitem{bib:vapid-rfc} M. Thomson, P. Beverloo, ``Voluntary Application Server Identification (VAPID)'', \url{https://datatracker.ietf.org/doc/html/rfc8292}, RFC8292, November 2017, \refdate{2024年1月21日}.
    
    \bibitem{bib:nginx-web} Nginx, ``nginx'', \url{https://nginx.org/en/}, \refdate{2024年1月18日}.
    \bibitem{bib:postform-web} Mozilla Foundation, ``POST - HTTP \textbar\ MDN'', \url{https://developer.mozilla.org/ja/docs/Web/HTTP/Methods/POST}, \refdate{2024年1月22日}.
    % \bibitem{bib:} , "", \url{https://developer.mozilla.org/ja/docs/Web/HTTP/Methods/POST}, \refdate{2024年1月22日}.
    
    
    % \bibitem{bib:Ishizaka}
    % 石坂守,山口賢一,岩田大志:``スキャン非同期記憶素子およびそれを備えた半導体集積回路ならびにその設計方法およびテストパターン'', 電子情報通信学会論文A,Vol.j102-A,No.6,pp.172-181,Jun.(2019)

    % \bibitem{textbook} \UTF{9AD9}玉 圭樹, "マルエージェント学習 -- 相互作用の謎に迫る --", 株式会社コロナ社, 2003年4月23日.
    % \bibitem{book} author, "title", publish, 2020.
    % \bibitem{webpage} author, "title", \url{URL}, \refdate{2021年8月15日}.
\end{thebibliography}

% なるべく書籍がいい
% 念の為の公式のテキストもメモ
% 
% * docker
% 本がある
% * docker-compose
% 本がある
% * Node.js
% * Leaflet.js
% https://leafletjs.com
% * PushAPI
% https://www.w3.org/TR/push-api/
% https://www.w3.org/standards/types/#WD
% https://developer.mozilla.org/en-US/docs/Web/API/Push_API
% * Nginx
% * Reverse Proxy

\expandafter\ifx\csname ifdraft\endcsname\relax
    % ----- 仕掛け 開始 -----
    \newcommand{\ifdraft}{false}
    % ----- 仕掛け 終了 -----
    % \include{references.tex}
    \end{document}
\fi
    \end{document}
\fi

% 謝辞
% \wordsofthanks{}
% %! Tex root = ../main.tex
\expandafter\ifx\csname ifdraft\endcsname\relax
    \documentclass[12pt]{honka_v1}
    \usepackage{soturon}
    \begin{document}
\fi

\wordsofthanks{
    本研究を行うにあたって
    親身かつ丁寧にご指導してくださりました,
    山口賢一教授,岩田大志准教授に最大限の感謝したします.
    
    研究を行う際に助言や相談に乗っていただいた,
    山口賢一研究室,岩田研究室の皆様,心よりお礼申し上げます.
    特に共に研究を行なった山口賢一研究室の山口璃桜さん,岩田研究室の松尾慧一さん
    には感謝の意を表します.
}

\expandafter\ifx\csname ifdraft\endcsname\relax
    % ----- 仕掛け 開始 -----
    \newcommand{\ifdraft}{false}
    % ----- 仕掛け 終了 -----
    %! Tex root = ../main.tex
\expandafter\ifx\csname ifdraft\endcsname\relax
    \documentclass[12pt]{honka_v1}
    \usepackage{soturon}
    \begin{document}
\fi
% 以下本文
% \printbibliography[title=参考文献]
% 参考文献をchapterとして目次に追加
% \addcontentsline{toc}{chapter}{参考文献}
\begin{thebibliography}{99}
    \bibitem{bib:jugaihigai-web} 農林水産省, ``野生鳥獣による農作物被害状況の推移'', \url{https://www.maff.go.jp/j/seisan/tyozyu/higai/hogai_zyoukyou/attach/pdf/index-31.pdf}, \refdate{2024年1月18日}.
    
    \bibitem{bib:docker-book} 大澤文孝, 浅井尚, ``触って学ぶクラウドインフラ docker 基礎からのコンテナ構築'', 日経BPマーケティング, 2020年.
    
    \bibitem{bib:node-book} 掌田津耶乃, ``Node.js 超入門'', 株式会社 秀和システム, 2017年.
    \bibitem{bib:express-web} StrongLoop, IBM, ``Express - Node.js web application framework'', \url{https://expressjs.com/}, \refdate{2024年1月18日}.
    \bibitem{bib:leafletjs-web} Volodymyr Agafonkin, ``Leaflet - a JavaScript library for interactive maps'', \url{https://leafletjs.com}, \refdate{2024年1月18日}.
    
    \bibitem{bib:pushapi-web} Mozilla Foundation, ``プッシュ API - Web API \textbar\ MDN'', \url{https://developer.mozilla.org/ja/docs/Web/API/Push_API}, \refdate{2024年1月18日}.
    \bibitem{bib:webpush-rfc} M. Thomson, E. Damaggio, B. Raymor, Ed., ``Generic Event Delivery Using HTTP Push'', \url{https://datatracker.ietf.org/doc/html/rfc8030}, RFC8030, December 2016, \refdate{2024年1月21日}.
    \bibitem{bib:serviceworker-web} Mozilla Foundation, ``サービスワーカー API - Web API \textbar\ MDN'', \url{https://developer.mozilla.org/ja/docs/Web/API/Service_Worker_API}, \refdate{2024年1月18日}.
    \bibitem{bib:vapid-rfc} M. Thomson, P. Beverloo, ``Voluntary Application Server Identification (VAPID)'', \url{https://datatracker.ietf.org/doc/html/rfc8292}, RFC8292, November 2017, \refdate{2024年1月21日}.
    
    \bibitem{bib:nginx-web} Nginx, ``nginx'', \url{https://nginx.org/en/}, \refdate{2024年1月18日}.
    \bibitem{bib:postform-web} Mozilla Foundation, ``POST - HTTP \textbar\ MDN'', \url{https://developer.mozilla.org/ja/docs/Web/HTTP/Methods/POST}, \refdate{2024年1月22日}.
    % \bibitem{bib:} , "", \url{https://developer.mozilla.org/ja/docs/Web/HTTP/Methods/POST}, \refdate{2024年1月22日}.
    
    
    % \bibitem{bib:Ishizaka}
    % 石坂守,山口賢一,岩田大志:``スキャン非同期記憶素子およびそれを備えた半導体集積回路ならびにその設計方法およびテストパターン'', 電子情報通信学会論文A,Vol.j102-A,No.6,pp.172-181,Jun.(2019)

    % \bibitem{textbook} \UTF{9AD9}玉 圭樹, "マルエージェント学習 -- 相互作用の謎に迫る --", 株式会社コロナ社, 2003年4月23日.
    % \bibitem{book} author, "title", publish, 2020.
    % \bibitem{webpage} author, "title", \url{URL}, \refdate{2021年8月15日}.
\end{thebibliography}

% なるべく書籍がいい
% 念の為の公式のテキストもメモ
% 
% * docker
% 本がある
% * docker-compose
% 本がある
% * Node.js
% * Leaflet.js
% https://leafletjs.com
% * PushAPI
% https://www.w3.org/TR/push-api/
% https://www.w3.org/standards/types/#WD
% https://developer.mozilla.org/en-US/docs/Web/API/Push_API
% * Nginx
% * Reverse Proxy

\expandafter\ifx\csname ifdraft\endcsname\relax
    % ----- 仕掛け 開始 -----
    \newcommand{\ifdraft}{false}
    % ----- 仕掛け 終了 -----
    % \include{references.tex}
    \end{document}
\fi
    \end{document}
\fi

%----------[ 参考文献 ]----------%
% \begin{thebibliography}{99}

%   \bibitem{bib:Ishizaka}
%   石坂守,山口賢一,岩田大志:``スキャン非同期記憶素子およびそれを備えた半導体集積回路ならびにその設計方法およびテストパターン'', 電子情報通信学会論文A,Vol.j102-A,No.6,pp.172-181,Jun.(2019)

% \end{thebibliography}
%! Tex root = ../main.tex
\expandafter\ifx\csname ifdraft\endcsname\relax
    \documentclass[12pt]{honka_v1}
    \usepackage{soturon}
    \begin{document}
\fi
% 以下本文
% \printbibliography[title=参考文献]
% 参考文献をchapterとして目次に追加
% \addcontentsline{toc}{chapter}{参考文献}
\begin{thebibliography}{99}
    \bibitem{bib:jugaihigai-web} 農林水産省, ``野生鳥獣による農作物被害状況の推移'', \url{https://www.maff.go.jp/j/seisan/tyozyu/higai/hogai_zyoukyou/attach/pdf/index-31.pdf}, \refdate{2024年1月18日}.
    
    \bibitem{bib:docker-book} 大澤文孝, 浅井尚, ``触って学ぶクラウドインフラ docker 基礎からのコンテナ構築'', 日経BPマーケティング, 2020年.
    
    \bibitem{bib:node-book} 掌田津耶乃, ``Node.js 超入門'', 株式会社 秀和システム, 2017年.
    \bibitem{bib:express-web} StrongLoop, IBM, ``Express - Node.js web application framework'', \url{https://expressjs.com/}, \refdate{2024年1月18日}.
    \bibitem{bib:leafletjs-web} Volodymyr Agafonkin, ``Leaflet - a JavaScript library for interactive maps'', \url{https://leafletjs.com}, \refdate{2024年1月18日}.
    
    \bibitem{bib:pushapi-web} Mozilla Foundation, ``プッシュ API - Web API \textbar\ MDN'', \url{https://developer.mozilla.org/ja/docs/Web/API/Push_API}, \refdate{2024年1月18日}.
    \bibitem{bib:webpush-rfc} M. Thomson, E. Damaggio, B. Raymor, Ed., ``Generic Event Delivery Using HTTP Push'', \url{https://datatracker.ietf.org/doc/html/rfc8030}, RFC8030, December 2016, \refdate{2024年1月21日}.
    \bibitem{bib:serviceworker-web} Mozilla Foundation, ``サービスワーカー API - Web API \textbar\ MDN'', \url{https://developer.mozilla.org/ja/docs/Web/API/Service_Worker_API}, \refdate{2024年1月18日}.
    \bibitem{bib:vapid-rfc} M. Thomson, P. Beverloo, ``Voluntary Application Server Identification (VAPID)'', \url{https://datatracker.ietf.org/doc/html/rfc8292}, RFC8292, November 2017, \refdate{2024年1月21日}.
    
    \bibitem{bib:nginx-web} Nginx, ``nginx'', \url{https://nginx.org/en/}, \refdate{2024年1月18日}.
    \bibitem{bib:postform-web} Mozilla Foundation, ``POST - HTTP \textbar\ MDN'', \url{https://developer.mozilla.org/ja/docs/Web/HTTP/Methods/POST}, \refdate{2024年1月22日}.
    % \bibitem{bib:} , "", \url{https://developer.mozilla.org/ja/docs/Web/HTTP/Methods/POST}, \refdate{2024年1月22日}.
    
    
    % \bibitem{bib:Ishizaka}
    % 石坂守,山口賢一,岩田大志:``スキャン非同期記憶素子およびそれを備えた半導体集積回路ならびにその設計方法およびテストパターン'', 電子情報通信学会論文A,Vol.j102-A,No.6,pp.172-181,Jun.(2019)

    % \bibitem{textbook} \UTF{9AD9}玉 圭樹, "マルエージェント学習 -- 相互作用の謎に迫る --", 株式会社コロナ社, 2003年4月23日.
    % \bibitem{book} author, "title", publish, 2020.
    % \bibitem{webpage} author, "title", \url{URL}, \refdate{2021年8月15日}.
\end{thebibliography}

% なるべく書籍がいい
% 念の為の公式のテキストもメモ
% 
% * docker
% 本がある
% * docker-compose
% 本がある
% * Node.js
% * Leaflet.js
% https://leafletjs.com
% * PushAPI
% https://www.w3.org/TR/push-api/
% https://www.w3.org/standards/types/#WD
% https://developer.mozilla.org/en-US/docs/Web/API/Push_API
% * Nginx
% * Reverse Proxy

\expandafter\ifx\csname ifdraft\endcsname\relax
    % ----- 仕掛け 開始 -----
    \newcommand{\ifdraft}{false}
    % ----- 仕掛け 終了 -----
    % \include{references.tex}
    \end{document}
\fi

%----------[ 付録 ]----------%


\end{document}
